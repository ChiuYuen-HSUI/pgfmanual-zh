% The titlepage

\newbox\mybox
{
  \parindent0pt
  \null
  \colorlet{mintgreen}{green!50!black!50}

  \thispagestyle{empty}
  \vskip3cm
  \vfill
  \hfil
  \begin{tikzpicture}[overlay]
    \coordinate (front) at (0,0);
    \coordinate (horizon) at (0,.31\paperheight);
    \coordinate (bottom) at (0,-.6\paperheight);
    \coordinate (sky) at (0,.57\paperheight);
    \coordinate (left) at (-.51\paperwidth,0);
    \coordinate (right) at (.51\paperwidth,0);

    \shade [bottom color=blue!30!black!10,top color=blue!30!black!50]
      ([yshift=-5mm]horizon -|  left) rectangle (sky -| right);
    \shade [bottom color=black!70!green!25,top color=black!70!green!10]
      (front -| left) -- (horizon -| left)
      decorate [decoration=random steps] { -- (horizon -| right) }
      -- (front -| right) -- cycle;
    \shade [top color=black!70!green!25,bottom color=black!25]
      ([yshift=-5mm-1pt]front -| left) rectangle ([yshift=1pt]front -| right);
    \fill [black!25] (bottom -| left) rectangle ([yshift=-5mm]front -| right);

    \def\nodeshadowed[#1]#2;{\node[scale=2,above,#1]{\global\setbox\mybox=\hbox{#2}\copy\mybox};
      \node[scale=2,above,#1,yscale=-1,scope fading=south,opacity=0.4]{\box\mybox};}

    \nodeshadowed [at={(-5,5  )},yslant=0.05] {\Huge Ti\textcolor{orange}{\emph{k}}Z};
    \nodeshadowed [at={( 0,5.3)}] {\huge \textcolor{mintgreen}{\&}};
    \nodeshadowed [at={( 5,5  )},yslant=-0.05] {\Huge \textsc{PGF}};
    % \nodeshadowed [at={( 0,2  )}] {Manual for Version \pgftypesetversion};
    \nodeshadowed [at={( 0,2  )}] {\tikzname\ \pgftypesetversion 中文手册};

    \foreach \where in {-9cm,9cm}
    {\nodeshadowed [at={(\where,5cm)}] {
    \tikz \draw [green!20!black, rotate=90]
    [l-system={rule set={F -> FF-[-F+F]+[+F-F]}, axiom=F, order=4,
      step=2pt, randomize step percent=50, angle=30, randomize angle percent=5}]
    lindenmayer system;};}

    \foreach \i in {0.5,0.6,...,2}
      \fill [white,decoration=Koch snowflake,opacity=.9]
            [shift=(horizon),shift={(rand*11,rnd*7)},scale=\i]
            [double copy shadow={opacity=0.2,shadow xshift=0pt,shadow
              yshift=3*\i pt,fill=white,draw=none}]
        decorate {
          decorate {
            decorate {
              (0,0) -- ++(60:1) -- ++(-60:1) -- cycle
            }
          }
        };

  \node (left text) [text width=.5\paperwidth-2cm,below right,at={(-.5\paperwidth+1cm,-1.5cm)}]
  {
    \fontfamily{pcr}
    \def\textbraceleft{\char`\{}
    \def\textbraceright{\char`\}}
    \def\textbackslash{\char`\\}
    \begin{lstlisting}[basicstyle=\scriptsize\color{black},
                       keywordstyle=\bfseries\color{white},
                       identifierstyle=\bfseries\color{black},
                       keywords={tikzpicture,shade,fill,draw,path,node},
                       literate={-}{{-}}1]
\begin{tikzpicture}
  \coordinate (front) at (0,0);
  \coordinate (horizon) at (0,.31\paperheight);
  \coordinate (bottom) at (0,-.6\paperheight);
  \coordinate (sky) at (0,.57\paperheight);
  \coordinate (left) at (-.51\paperwidth,0);
  \coordinate (right) at (.51\paperwidth,0);

  \shade [bottom color=white,
          top color=blue!30!black!50]
              ([yshift=-5mm]horizon -|  left)
    rectangle (sky -| right);

  \shade [bottom color=black!70!green!25,
          top color=black!70!green!10]
    (front -| left) -- (horizon -| left)
    decorate [decoration=random steps] {
      -- (horizon -| right)  }
    -- (front -| right) -- cycle;

  \shade [top color=black!70!green!25,
         bottom color=black!25]
              ([yshift=-5mm-1pt]front -| left)
    rectangle ([yshift=1pt]front -| right);

  \fill [black!25]
              (bottom -| left)
    rectangle ([yshift=-5mm]front -| right);

  \def\nodeshadowed[#1]#2;{
    \node[scale=2,above,#1]{
      \global\setbox\mybox=\hbox{#2}
      \copy\mybox};
    \node[scale=2,above,#1,yscale=-1,
          scope fading=south,opacity=0.4]{\box\mybox};
  }
\end{lstlisting}
};

  \node (right text) [text width=.5\paperwidth-2cm,below right,at={(1cm,-1.5cm)}]
  {
    \fontfamily{pcr}
    \def\textbraceleft{\char`\{}
    \def\textbraceright{\char`\}}
    \def\textbackslash{\char`\\}
    \begin{lstlisting}[basicstyle=\scriptsize\color{black},
                       keywordstyle=\bfseries\color{white},
                       identifierstyle=\bfseries\color{black},
                       keywords={tikzpicture,shade,fill,draw,path,node},
                       literate={-}{{-}}1]
  \nodeshadowed [at={(-5,8  )},yslant=0.05]
    {\Huge Ti\textcolor{orange}{\emph{k}}Z};
  \nodeshadowed [at={( 0,8.3)}]
    {\huge \textcolor{green!50!black!50}{\&}};
  \nodeshadowed [at={( 5,8  )},yslant=-0.05]
    {\Huge \textsc{PGF}};
  \nodeshadowed [at={( 0,5  )}]
    {Manual for Version \pgftypesetversion};

  \foreach \where in {-9cm,9cm} {
    \nodeshadowed [at={(\where,5cm)}] { \tikz
      \draw [green!20!black, rotate=90,
             l-system={rule set={F -> FF-[-F+F]+[+F-F]},
               axiom=F, order=4,step=2pt,
               randomize step percent=50, angle=30,
               randomize angle percent=5}] l-system; }}

  \foreach \i in {0.5,0.6,...,2}
    \fill
      [white,opacity=\i/2,
       decoration=Koch snowflake,
       shift=(horizon),shift={(rand*11,rnd*7)},
       scale=\i,double copy shadow={
         opacity=0.2,shadow xshift=0pt,
         shadow yshift=3*\i pt,fill=white,draw=none}]
      decorate {
        decorate {
          decorate {
            (0,0)- ++(60:1) -- ++(-60:1) -- cycle
          } } };

   \node (left text) ...
   \node (right text) ...

   \fill [decorate,decoration={footprints,foot of=gnome},
          opacity=.5,brown]        (rand*8,-rnd*10)
     to [out=rand*180,in=rand*180] (rand*8,-rnd*10);
\end{tikzpicture}
  \end{lstlisting}
  };

  \fill [decorate,decoration=footprints,
         decoration={footprints,foot of=gnome},
         opacity=.5,brown]        (rand*8,-rnd*10)
    to [out=rand*180,in=rand*180] (rand*8,-rnd*10);
\end{tikzpicture}
\vfill
\vbox{}
\clearpage
}

{
  \vbox{}
  \vskip0pt plus 1fill
  F\"ur meinen Vater, damit er noch viele sch\"one \TeX-Graphiken
  erschaffen kann.
  \vskip1em
  \hfill\emph{Till}
  \vskip0pt plus 3fill

  \parindent=0pt
  Copyright 2007 to 2013 by Till Tantau

  \medskip
  Permission is granted to copy, distribute and/or modify \emph{the documentation}
  under the terms of the \textsc{gnu} Free Documentation License, Version 1.2
  or any later version published by the Free Software Foundation;
  with no Invariant Sections, no Front-Cover Texts, and no Back-Cover Texts.
  A copy of the license is included in the section entitled \textsc{gnu}
  Free Documentation License.

  \medskip
  Permission is granted to copy, distribute and/or modify \emph{the
    code of the package} under the terms of the \textsc{gnu} Public License, Version 2
  or any later version published by the Free Software Foundation.
  A copy of the license is included in the section entitled \textsc{gnu}
  Public License.

  \medskip
  Permission is also granted to distribute and/or modify \emph{both
    the documentation and the code} under the conditions of the LaTeX
  Project Public License, either version 1.3 of this license or (at
  your option) any later version. A copy of the license is included in
  the section entitled \LaTeX\ Project Public License.

  \vbox{}
  \clearpage
}

% \title{\bfseries The \tikzname\ and {\Large PGF} Packages\\
%   \large Manual for version \pgfversion\\[1mm]
% \large\href{http://sourceforge.net/projects/pgf}{\texttt{http://sourceforge.net/projects/pgf}}}
\title{\bfseries \tikzname\ 和 {\Large PGF} 宏包\\
  \large \tikzname\ \pgfversion 中文手册\footnote{{\color{blue}翻译者:Hansimov。此项目开源在 GitHub,欢迎提出建议或参与翻译:https://github.com/Hansimov/pgfmanual-zh}}\\[1mm]
\large\href{http://sourceforge.net/projects/pgf}{\texttt{http://sourceforge.net/projects/pgf}}}

% \author{Till Tantau\footnote{Editor of this documentation. Parts of
%     this documentation have been written by other authors as indicated
%     in these parts or chapters and in Section~\ref{section-authors}.}\\
\author{Till Tantau\footnote{该手册的编者。手册的部分内容由另外一些作者写成,详见~\ref{section-authors}~小节。}\\
  \normalsize Institut f\"ur Theoretische Informatik\\[-1mm]
  \normalsize Universit\"at zu L\"ubeck}

\date{August 29, 2015 \footnote{{\color{blue}中文手册更新时间:\the\year 年 \the\month 月 \the\day 日}}}

\maketitle

\tableofcontents

\clearpage