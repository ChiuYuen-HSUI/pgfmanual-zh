% Copyright 2008 by Till Tantau
%
% This file may be distributed and/or modified
%
% 1. under the LaTeX Project Public License and/or
% 2. under the GNU Free Documentation License.
%
% See the file doc/generic/pgf/licenses/LICENSE for more details.



\section{Object-Oriented Programming}

\label{section-oop}

This section describes the |oo| module.

\begin{pgfmodule}{oo}
  This module defines a relatively small set of \TeX\ commands for
  defining classes, methods, attributes and objects in the sense of
  object-oriented programming.
\end{pgfmodule}

In this chapter it is assumed that you are familiar with the basics
of a typical object-oriented programming language like Java, C++ or
Eiffel.



\subsection{Overview}

\TeX\ does not support object-oriented programming, presumably because
it was written at a time when this style of programming was not yet
``en vogue.'' When one is used to the object-oriented style of
thinking, some programming constructs in \TeX\ often seem overly
complicated. The object-oriented programming module of \pgfname\ may
help here. It is written completely using simple \TeX\ macros and is,
thus, perfectly portable. This also means, however, that it is not
particularly fast (but not too slow either), so you should use it only
for non-time-critical things.

Basically, the oo-system supports \emph{classes} (in the
object-oriented sense, this has nothing to do with \LaTeX-classes),
\emph{methods}, \emph{constructors}, \emph{attributes},
\emph{objects}, \emph{object identities}, and (thanks to Sa\v o \v
Zivanovi\'c) \emph{inheritance} and \emph{overloading.} 

The first step is to define a class, using the macro |\pgfooclass|
(all normal macros in \pgfname's object-oriented system start with
|\pgfoo|). This macro gets the name of a class and in its body a
number of \emph{methods} are defined. These are defined using the
|\method| macro (which is defined only inside such a class definition)
and they look a bit like method definitions in, say, Java. Object
attributes are declared using the |\attribute| command, which is also
defined only inside a class definition.

Once a class has been defined, you can create objects of this
class. Objects are created using |\pgfoonew|. Such an object has many
characteristics of objects in a normal object-oriented programming
language: Each object has a \emph{unique identity}, so when you create
another object, this object is completely distinct from all other
objects. Each object also has a set of private attributes, which may
change over time. Suppose, for instance, that we have a |point|
class. Then creating a new object (called an instance) of this class
would typically have an |x|-attribute and a |y|-attribute. These can
be changed over time. Creating another instance of the
|point| class creates another object with its own |x|- and
|y|-attributes.

Given an object, you can call a method for this object. Inside the
method the attributes of the object for which the method is being
called can be accessed.

The life of an object always ends with the end of the \TeX\ scope in
which it was created. However, changes to attribute values are not
local to scopes, so when you change an attribute anywhere, this change
persists till the end of the life of the object or until the attribute
is changed again.

\subsection{A Running Example: The Stamp Class}

As a running example we will develop a |stamp| class and |stamp|
objects. The idea is that a stamp object is able to ``stamp
something'' on a picture. This means that a stamp object has an
attribute storing the ``stamp text'' and there is a method that asks
the object to place this text somewhere on a
canvas. The method can be called repeatedly and there can be several
different stamp objects, each producing a different text. Stamp
objects can either be created dynamically when needed or a library
might define many such objects in an outer scope.

Such stamps are similar to many things present in \pgfname\ such as
arrow tips, patterns, or shadings and, indeed, these could all have
been implemented in this object-oriented fashion (which might have
been better, but the object-oriented subsystem is a fairly new
addition to \pgfname).


\subsection{Classes}

We start with the definition of the |stamp| class. This is done using
the |\pgfooclass| macro:

\begin{command}{\pgfooclass\opt{|(|\meta{list of superclasses}|)|}\marg{class name}\marg{body}}
  This command defines a class named \meta{class name}. The name of
  the class can contain spaces and most other characters, but no
  periods. So, valid class names are |MyClass| or |my class| or
  |Class_C++_emulation??1|. The \meta{list of superclasses} is
  optional just like the parenthesis around it.

  The \meta{body} is actually just executed, so any
  normal \TeX-code is permissible here. However, while the \meta{body} is
  being executed, the macros |\method| and |\attribute| are set up so
  that they can be used to define methods and attributes for this
  class (the original meanings are restored afterward).

  The definition of a class is local to the scope where the class has
  been defined.
\begin{codeexample}[code only]
\pgfooclass{stamp}{
  % This is the class stamp

  \attribute text;
  \attribute rotation angle=20;

  \method stamp(#1) { % The constructor
    ...
  }

  \method apply(#1,#2) { % Causes the stamp to be shown at coordinate (#1,#2)
    ...
  }
}

% We can now create objects of type "stamp"
\end{codeexample}

  Concerning the list of base classes, the Method Resolution Order
  (\textsc{mro}) is computed using the C3 algorithm also used in Python, v2.3
  and higher. The linearization computed by the algorithm respects
  both  local precedence ordering and monotonicity. Resolution of both
  methods and attributes depends on the \textsc{mro}: when a method method name
  is called on an object of class $C$, the system invokes method
  method name from the first class in the \textsc{mro} of $C$ which defines
  method method name; when an object is created, each attribute |attr|
  is initialized to the value specified in the first class in the \textsc{mro}
  of $C$ which declares attribute |attr|. 
\end{command}

The \meta{body} of a class usually just consists of calls to the macros
|\attribute| and |\method|, which will be discussed in more detail in
later sections.



\subsection{Objects}

Once a class has been declared, we can start creating objects for this
class. For this the |\pgfoonew| command can be used, which has a
peculiar syntax:

\begin{command}{\pgfoonew\opt{\meta{object handle or attribute}|=|}|new |\meta{class
      name}|(|\meta{constructor arguments}|)|}
  Causes a new object to be created. The class of the object will be
  \meta{class name}, which must previously have been declared using
  |\pgfooclass|. Once the object has been created, the constructor
  method of the object will be called with the parameter list set to
  \meta{constructor arguments}.

  The resulting object is stored internally and its lifetime will
  end exactly at the end of the current scope.

  Here is an example in which three stamp objects are created.
\begin{codeexample}[code only]
\pgfoonew \firststamp  = new stamp()
\pgfoonew \secondstamp = new stamp()
{
  \pgfoonew \thirdstamp = new stamp()
  ...
}
% \thirdstamp no longer exists, but \firststamp and \secondstamp do
% even if you try to store \thirdstamp in a global variable, trying
% to access it will result in an error.
\end{codeexample}

  The optional \meta{object handle or attribute} can either be an
  \meta{object handle} or an \meta{attribute}. When an \meta{object
    handle} is given, it must be a normal \TeX\ macro name that will
  ``point'' to the object (handles are discussed in more detail in
  Section~\ref{section-identities}). You can use this macro to call
  methods of the object as discussed in the following section. When an
  \meta{attribute} is given, it must be given in curly braces (the
  curly braces are used to detect the presence of an attribute). In
  this case, a handle to the newly created object is stored in this
  attribute.
\begin{codeexample}[code only]
\pgfooclass{foo}
{
  \attribute stamp obj;
  \attribute another object;

  \method \foo() {
    \pgfoonew{stamp obj}=new stamp()
    \pgfoonew{another object}=new bar()
  }
  ...
}
\end{codeexample}
\end{command}

\begin{command}{\pgfoogc}
  This command causes the ``garbage collector'' to be invoked. The job
  of this garbage collector is to free the global \TeX-macros that are
  used by ``dead'' objects (objects whose life-time has ended). This
  macro is called automatically after every scope in which an object
  has been created, so you normally do not need to call this macro
  yourself.
\end{command}


\subsection{Methods}

Methods are defined inside the body of classes using the following
command:

\begin{command}{\method \meta{method name}|(|\meta{parameter list}|)|\marg{method body}}
  This macro, which is only defined inside a class definition, defines
  a new method named \meta{method name}. Just like class names, method
  names can contain spaces and other characters, so \meta{method names} like
  |put_stamp_here| or |put stamp here| are both legal.

  Three method names are special: First, a method having either the
  same name as the class or having the name |init| is called the
  \emph{constructor} of the class. There are (currently) no
  destructors; objects simply become ``undefined'' at the
  end of the scope in which they have been created. The other two
  methods are called |get id| and |get handle|, which are always automatically
  defined and which you cannot redefine. They are discussed in
  Section~\ref{section-identities}.

  Overloading of methods by differing numbers of parameters is not
  possible, that is, it is illegal to have two methods inside a single
  class with the same name (despite possibly different parameter
  lists). However, two different classes may contain a method with the
  same name, that is, classes form namespaces for methods. Also, a
  class can (re)implement a method from a superclass.

  The \meta{method name} must be followed by a \meta{parameter list}
  in parentheses, which must be present even when the \meta{parameter
    list} is empty. The \meta{parameter list} is actually a normal
  \TeX\ parameter list that will be matched against the parameters
  inside the parentheses upon method invocation and, thus, could be
  something like |#1#2 foo #3 bar.|, but a list like |#1,#2,#3| is
  more customary. By setting the parameter list to just |#1| and then
  calling, say,  |\pgfkeys{#1}| at the beginning of a method, you can
  implement Objective-C-like named parameters.

  When a method is called, the \meta{body} of
  the method will be executed. The main difference to a normal macro
  is that while the \meta{body} is executed, a special macro called
  |\pgfoothis| is set up in such a way that it references the object
  for which the method is executed.
\end{command}

In order to call a method for an object, you first need to create the
object and you need a handle for this object. In order to invoke a
method for this object, a special syntax is used that is similar to
Java or C++ syntax:

\begin{pgfmanualentry}
  \pgfmanualentryheadline{\meta{object handle}\opt{|.|\meta{super
        class}}|.|\meta{method name}|(|\meta{parameters}|)|}% 
  \pgfmanualbody
  This causes the method \meta{method name} to be called for the
  object referenced by the \meta{object handle}. The method is the one
  defined in the class of the object or, if it is not defined there,
  the method defined in the superclasses of the object's class (if
  there are several superclasses that define the same method, the
  method resolution order is used to determine which one gets
  called). If the optional \meta{super class} is specified, the method
  implementation of that class will be used rather than the
  implementation in the object's class. The \meta{parameters} 
  are matched against the parameters of the method and, then, the
  method body is executed. The execution of the method body is
  \emph{not} done inside a scope, so the effects of a method body
  persist.

\begin{codeexample}[code only]
\pgfooclass{stamp}{
  % This is the class stamp

  \method stamp() { % The constructor
  }

  \method apply(#1,#2) { % Causes the stamp to be shown at coordinate (#1,#2)
    % Draw the stamp:
    \node [rotate=20,font=\huge] at (#1,#2) {Passed};
  }
}

\pgfoonew \mystamp=new stamp()

\begin{tikzpicture}
  \mystamp.apply(1,2)
  \mystamp.apply(3,4)
\end{tikzpicture}
\end{codeexample}

  Inside a method, you can call other methods. If you have a handle
  for another object, you can simply call it in the manner described
  above. In order to call a method of the current object, you can use
  the special object handle |\pgfoothis|.

  \begin{command}{\pgfoothis}
    This object handle is defined only when a method is being
    executed. There, it is then set to point to the object for which
    the method is being called, which allows you to call another
    method for the same object.

\begin{codeexample}[code only]
\pgfooclass{stamp}{
  % This is the class stamp

  \method stamp() {}

  \method apply(#1,#2) {
    \pgfoothis.shift origin(#1,#2)

    % Draw the stamp:
    \node [rotate=20,font=\huge] {Passed};
  }

  % Private method:
  \method shift origin(#1,#2) {
    \tikzset{xshift=#1,yshift=#2}
  }
}
\end{codeexample}
  \end{command}
\end{pgfmanualentry}


\begin{command}{\pgfoosuper|(|\meta{class},\meta{object
      handle}|).|\meta{method name}|(|\meta{arguments}|)|}
  This macro gives you finer control over which method gets invoked in
  case of multiple inheritance. This macro calls \meta{method name} of
  the object specified by \meta{object handle}, but which
  implementation of the method is called is determined as follows: it
  will be the implementation in the first class (in the method
  resolution order) after \meta{class} that defines \meta{method
    name}. 
\end{command}

\subsection{Attributes}

Every object has a set of attributes, which may change over
time. Attributes are declared using the |\attribute| command, which,
like the |\method| command, is defined only inside the scope of
|\pgfooclass|. Attributes can be modified (only) by methods. To take
the |stamp| example, an attribute of a |stamp| object might be the
text that should be stamped when the |apply| method is called.

When an attribute is changed, this change is \emph{not} local to the
current \TeX\ group. Changes will persist till the end
of the object's life or until the attribute is changed once more.

To declare an attribute you should use the |\attribute| command:
\begin{command}{\attribute \meta{attribute name}\opt{|=|\meta{initial
        value}}|;|}
  This command can only be given inside the body of an |\pgfooclass|
  command. It declares the attribute named \meta{attribute name}. This
  name, like method or class names, can be quite arbitrary, but should
  not contain periods. Valid names are |an_attribute?| or
  |my attribute|.

  You can optionally specify an \meta{initial value} for the
  attribute; if none is given, the empty string is used
  automatically. The initial value is the value that the attribute
  will have just after the object has been created and before the
  constructor is called.

\begin{codeexample}[code only]
\pgfooclass{stamp}{
  % This is the class stamp

  \attribute text;
  \attribute rotation angle = 20;

  \method stamp(#1) {
    \pgfooset{text}{#1} % Set the text
  }

  \method apply(#1,#2) {
    \pgfoothis.shift origin(#1,#2)

    % Draw the stamp:
    \node [rotate=\pgfoovalueof{rotation angle},font=\huge]
      {\pgfoovalueof{text}};
  }

  \method shift origin(#1,#2) { ... }

  \method set rotation (#1) {
    \pgfooset{rotation angle}{#1}
  }
}
\end{codeexample}
\end{command}

Attributes can be set and read only inside methods, it is not possible
to do so using an object handle. Spoken in terms of traditional
object-oriented programming, attributes are always private. You need
to define getter and setter methods if you wish to read or modify
attributes.

Reading and writing attributes is not done using the ``dot-notation''
that is used for method calls. This is mostly due to efficiency
reasons. Instead, a set of special macros is used, all of which can
\emph{only be used inside methods}.

\begin{command}{\pgfooset\marg{attribute}\marg{value}}
  Sets the \meta{attribute} of the current object to
  \meta{value}.
\begin{codeexample}[code only]
\method set rotation (#1) {
  \pgfooset{rotation angle}{#1}
}
\end{codeexample}
\end{command}

\begin{command}{\pgfoolet\marg{attribute}\marg{macro}}
  Sets the \meta{attribute} of the current value to the current value
  of \meta{macro} using \TeX's |\let| command.
\begin{codeexample}[code only]
\method foo () {
  \pgfoolet{my func}\myfunc
  % Changing \myfunc now has no effect on the value of attribute my func
}
\end{codeexample}
\end{command}

\begin{command}{\pgfoovalueof\marg{attribute}}
  Expands  (eventually) to the current value of \meta{attribute} of
  the current object.
\begin{codeexample}[code only]
\method apply(#1,#2) {
  \pgfoothis.shift origin(#1,#2)

  \node [rotate=\pgfoovalueof{rotation angle},font=\huge]
    {\pgfoovalueof{text}};
}
\end{codeexample}
\end{command}

\begin{command}{\pgfooget\marg{attribute}\marg{macro}}
  Reads the current value of \meta{attribute} and stores the result in
  \meta{macro}.
\begin{codeexample}[code only]
...
  \method get rotation (#1) {
    \pgfooget{rotation angle}{#1}
  }
...

\mystamp.get rotation(\therotation)
``\therotation'' is now ``20'' (or whatever).
\end{codeexample}
\end{command}


\subsection{Identities}
\label{section-identities}

Every object has a unique identity, which is simply an integer. It is
possible to retrieve the object id using the |get id| method (discussed
below), but normally you will not need to do so because the id itself
cannot be used to access an object. Rather, you access objects via
their methods and these, in turn, can only be called via object
handles.

Object handles can be created in four ways:
\begin{enumerate}
\item Calling |\pgfoonew|\meta{object handle}|=...| will cause
  \meta{object handle} to be a handle to the newly created object.
\item Using |\let| to create an alias of an existing object handle: If
  |\mystamp| is a handle, saying |\let\myotherstamp=\mystamp| creates
  a second handle to the same object.
\item |\pgfooobj|\marg{id} can be used as an object handle to the
  object with the given \meta{id}.
\item Using the |get handle| method to create a handle to a given object.
\end{enumerate}

Let us have a look at the last two methods.

\begin{command}{\pgfooobj\marg{id}}
  Provided that \meta{id} is the id of an existing object (an object
  whose life-time has not expired), calling this command yields a
  handle to this object. The handle can then be used to call methods:
\begin{codeexample}[code only]
% Create a new object:
\pgfoonew \mystamp=new stamp()

% Get the object's id and store it in \myid:
\mystamp.get id(\myid)

% The following two calls have the same effect:
\mystamp.apply(1,1)
\pgfooobj{\myid}.apply(1,1)
\end{codeexample}
\end{command}

The |get id| method can be used to retrieve the id of an object. This
method is predefined for every class and you should not try to define
a method of this name yourself.

\begin{predefinedmethod}{get id(\meta{macro})}
  Calling \meta{obj}|.get id(|\meta{macro}|)|  stores the id \meta{obj} in
  \meta{macro}. This is mainly useful when you wish to store an
  object for a longer time and you cannot guarantee that any handle
  that you happen to have for this object will be available later on.

  The only way to use the retrieved id later on is to call
  |\pgfooobj|.

  Different object that are alive (that are still within the scope in
  which they were created) will always have different ids, so you can
  use the id to test for equality of objects. However, after an object
  has been destroyed because its scope has ended, the same id may be
  used again for newly created objects.

  Here is a typical application where you need to call this method:
  You wish to collect a list of objects for which you wish to call a
  specific method from time to time. For the collection process you
  wish to offer a macro called |\addtoobjectlist|, which takes an
  object handle as parameter. It is quite easy to store this handle
  somewhere, but a handle is, well, just a handle. Typically, shortly
  after the call to |\addtoobjectlist| the handle will no longer be
  valid or even exist, even though the object still exists. In this
  case, you wish to store the object id somewhere instead of the
  handle. Thus, for the object passed to |\addtoobjectlist| you call
  the |get id| method and store the resulting id, rather than the handle.
\end{predefinedmethod}

There is a second predefined method, called |get handle|, which is also
used to create object handles.

\begin{predefinedmethod}{get handle(\marg{macro name})}
  Calling this method for an object will cause \meta{macro name} to
  become a handle to the given object. For any object handle |\obj| --
  other than |\pgfoothis| -- the following two have the same effect:
  \begin{enumerate}
  \item |\let|\meta{macro name}|=\obj|
  \item |\obj.get handle(|\meta{macro name}|)|
  \end{enumerate}

  The first method is simpler and faster. However, for |\pgfoothis|
  there is a difference: The call |\pgfoothis.get handle(|\meta{macro name}|)| will
  cause \meta{macro name} to be an object handle to the current
  object and will persist to be so even after the method is done. By
  comparison, |\let|\meta{macro name}|=\pgfoothis| causes |\obj| to be
  the same as the very special macro |\pgfoothis|, so |\obj| will
  always refer to the current object, which may change over time.
\end{predefinedmethod}




\subsection{The Object Class}

\label{section-object}

The object-oriented module predefines a basic class |object| that can
be used as a base class in different context.

\begin{ooclass}{object}
  This class current only implements one method:

  \begin{method}{copy(\meta{handle})}
    Creates a new object and initializes the values of its (declared)
    attributes to the values of the original. The method takes one
    argument: a control sequence which receives the handle of the
    copy. 
  \end{method}
\end{ooclass}


\subsection{The Signal Class}

\label{section-signals}

In addition to the basic mechanism for defining and using classes and
object, the class |signal| is predefined. It implements a so-called
signal--slot mechanism. 

\begin{ooclass}{signal}
  This class is used to implement a simple signal--slot
  mechanism. The idea is the following: From time to time special
  things happen about which a number of objects need to be
  informed. Different things can happen and different object will be
  interested in these things. A |signal| object can be used to signal
  that such special things of a certain kind have happened. For
  example, one signal object might be used to signal the event that ``a
  page has been shipped out.'' Another signal might be used to signal
  that ``a figure is about to be typeset,'' and so on.

  Objects can ``tune in'' to signals. They do so by \emph{connecting}
  one of their methods (then called a \emph{slot}) to the
  signal. Then, whenever the signal is \emph{emitted}, the method of
  the connected object(s) get called. Different objects can connect
  different slots to the same signal as long as the argument lists
  will fit. For example, the object that is used to signal the ``end
  of page has been reached'' might emit signals that have, say, the
  box number in which the finished page can be found as a parameter
  (actually, the finished page is always in box 255). Then one object
  could connect a method |handle page(#1)| to this signal, another
  might connect the method |emergency action(#1)| to this signal, and
  so on.

  Currently, it is not possible to ``unregister'' or ``detach'' a slot
  from a signal, that is, once an object has been connect to a signal,
  it will continue to receive emissions of this signal till the end
  of the life-time of the signal. This is even true when the object no
  longer exists (but the signal does), so care must be taken that
  signal objects are always created before the objects that are
  listening to them.

  \begin{method}{signal()}
    The constructor does nothing.
  \end{method}

  \begin{method}{connect(\meta{object handle},\meta{method name})}
    This method gets an \meta{object handle} as parameter and a
    \meta{method name} of this object. It will queue the object-method
    pair in an internal list and each time the signal emits something,
    this object's method is called.

    Be careful not to pass |\pgfoothis| as \meta{object handle}. This
    would cause the signal object to connect to itself. Rather, if you
    wish to connect a signal to a method of the current object you
    first need to create an alias using the |get handle| method:
\begin{codeexample}[code only]
\pgfooclass{some class}{
  \method some class() {
    \pgfoothis.get handle(\me)
    \somesignal.connect(\me,foo)
    \anothersignal.connect(\me,bar)
  }
  \method foo () {}
  \method bar (#1,#2) {}
}
\pgfoonew \objA=new some class()
\pgfoonew \objB=new some class()
\end{codeexample}
  \end{method}

  \begin{method}{emit(\meta{arguments})}
    This method emits a signal to all connected slots. This means that
    for all objects that have previously been connected via a call of
    |connect|, the method (slot) that was specified during the call of
    |connect| is invoked with given \meta{arguments}.
\begin{codeexample}[code only]
\anothersignal.emit(1,2)
% will call \objA.bar(1,2) and \objB.bar(1,2)
\end{codeexample}
  \end{method}
\end{ooclass}


\subsection{Implementation Notes}

For the curious, here are some notes on how the oo-system is
implemented:

\begin{itemize}
\item There is an object id counter that gets incremented each time an
  object is created. However, this counter is local to the current
  scope, which means that it is reset at the end of each scope,
  corresponding to the fact that at the end of a scope all objects
  created in this scope become invalid. Newly created objects will
  then have the same id as ``deleted'' objects.
\item Attributes are stored globally. For each attribute of each
  object there is a macro whose name is composed of the object's id
  and the attribute name. Changes to object attributes are always
  global.
\item A call to the garbage collector causes a loop to be executed
  that tries to find objects whose object number is larger than the
  current maximum alive objects. The global attributes of these
  objects are then freed (set to |\relax|) by calling a special
  internal method of these (dead) objects.

  The garbage collector is automatically called after each group in
  which an object was created using |\aftergroup|.
\item When a method is called, before the method call some code is
  executed that sets a global counter storing the current object id to
  the object id of the object being called. After the method call some
  code is inserted that restores the global counter to its original
  value. This is done without scopes, so some tricky |\expandafter|
  magic is needed. Note that, because of this process, you cannot use
  commands like |\pgfutil@ifnextchar| at the end of a method.
\item An object handle contains just the code to set up and restore the
  current object number to the number of the object being called.
\end{itemize}
