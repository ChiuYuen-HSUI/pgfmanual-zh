% Copyright 2006 by Till Tantau
%
% This file may be distributed and/or modified
%
% 1. under the LaTeX Project Public License and/or
% 2. under the GNU Free Documentation License.
%
% See the file doc/generic/pgf/licenses/LICENSE for more details.

\section{Arrows}
\label{section-tikz-arrows}


\subsection{Overview}

\tikzname\ allows you to add (multiple) arrow tips to the end of
lines as in \tikz [baseline] \draw [->>]%<<
(0,.5ex) -- (3ex,.5ex); or in \tikz [baseline] \draw [-{Latex[]}]
(0,.5ex) -- (3ex,.5ex);. It is possible to change which arrow tips are
used ``on-the-fly,'' you can have several arrow tips in a row, and you
can change the appearance of each of them individually using a special
syntax. The following example is a perhaps slightly ``excessive''
demonstration of what you can do (you need to load the |arrows.meta|
library for it to work):
\begin{codeexample}[]
\tikz {
  \node [circle,draw] (A)              {A};
  \node [circle,draw] (B) [right=of A] {B};

  \draw [draw = blue, thick,
         arrows={
           Computer Modern Rightarrow [sep] 
         - Latex[blue!50,length=8pt,bend,line width=0pt]
           Stealth[length=8pt,open,bend,sep]}]
    (A) edge [bend left=45] (B)
    (B) edge [in=-110, out=-70,looseness=8] (B);
}
\end{codeexample}

There are a number of predefined generic arrow tip kinds whose
appearance you can modify in many ways using various options. It is
also possible to define completely new arrow tip kinds, see
Section~\ref{section-arrows}, but doing this is somewhat harder than
configuring an existing kind (it is like the difference between using
a font at different sizes or faces like italics, compared to
designing a new font yourself).

In the present section, we go over the various ways in which you can
configure which particular arrow tips are \emph{used.} The glorious
details of how new arrow tips can be defined are explained in
Section~\ref{section-arrows}.

At the end of the present section, Section~\ref{section-arrows-meta},
you will find a description of the different predefined arrow tips
from the |arrows.meta| library.

\emph{Remark:} Almost all of the features described in the following
were introduced in version 3.0 of \tikzname. For compatibility
reasons, the old arrow tips are still available. To differentiate
between the old and new arrow tips, the following rule is used: The
new, more powerful arrow tips start with an uppercase letter as in
|Latex|, compared to the old arrow tip |latex|.

\emph{Remark:} The libraries |arrows| and |arrows.spaced| are
deprecated. Use |arrows.meta| instead/additionally, which allows you
to do all that the old libraries offered, plus much more. However, the
old libraries still work and you can even mix old and new arrow tips
(only, the old arrow tips cannot be configured in the ways described
in the rest of this section; saying |scale=2| for a |latex| arrow has
no effect for instance, while for |Latex| arrows it doubles their size
as one would expect.)


\subsection{Where and When Arrow Tips Are Placed}
\label{section-arrow-tips-where}

In order to add arrow tips to the lines you draw, the following
conditions must be met:

\begin{enumerate}
\item You have specified that arrow tips should be added to 
  lines, using the |arrows| key or its short form.
\item You set the |tips| key to some value that causes tips to be
  drawn (to be explained later). 
\item You do not use the |clip| key (directly or indirectly) with the
  current path. 
\item The path actually has two ``end points'' (it is not
  ``closed'').
\end{enumerate}

Let us start with an introduction to the basics of the |arrows| key: 

\begin{key}{/tikz/arrows=\meta{start arrow specification}|-|\meta{end
      arrow specification}} 
  This option sets the arrow tip(s) to be used at the start and end of
  lines. An empty value as in |->| for the start indicates that no
  arrow tip should be drawn at the start.% 
  \indexoption{arrows}

  \emph{Note: Since the arrow option is so often used, you can leave
    out the text |arrows=|.} What happens is that every (otherwise
  unknown) option that contains a |-| is interpreted as an arrow specification.

\begin{codeexample}[]
\begin{tikzpicture}
  \draw[->]        (0,0)   -- (1,0);
  \draw[>-Stealth] (0,0.3) -- (1,0.3);
\end{tikzpicture}
\end{codeexample}

  In the above example, the first start specification is empty and the
  second is |>|. The end specifications are |>| for the first line and
  |Stealth| for the second line. Note that it makes a difference
  whether |>| is used in a start specification or in an end
  specification: In an end specification it creates, as one would
  expect, a pointed tip  at the end of the line. In the start
  specification, however, it creates a ``reversed'' version if this
  arrow -- which happens to be what one would expect here.

  The above specifications are very simple and only select a single
  arrow tip without any special configuration options, resulting in
  the ``natural'' versions of these arrow tips. It is also possible to
  ``configure'' arrow tips in many different ways, as explained in
  detail in Section~\ref{section-arrow-config} below by adding options
  in square brackets following the arrow tip kind:

\begin{codeexample}[]
\begin{tikzpicture}
  \draw[-{Stealth[red]}] (0,0)   -- (1,0);
\end{tikzpicture}
\end{codeexample}

  Note that in the example I had to surround the end specification by
  braces. This is necessary so that \tikzname\ does not mistake the
  closing square bracket of the |Stealth| arrow tip's options for the
  end of the options of the |\draw| command. In general, you often
  need to add braces when specifying arrow tips except for simple case
  like |->| or |<<->|, which are pretty frequent, though. When in
  doubt, say |arrows={|\meta{start spec}|-|\meta{end spec}|}|, that
  will always work.

  It is also possible to specify multiple (different) arrow tips in a
  row inside a specification, see Section~\ref{section-arrow-spec}
  below for details.
\end{key}

As was pointed out earlier, to add arrow tips to a path, the path must
have ``end points'' and not be ``closed'' -- otherwise adding arrow
tips makes little sense, after all. However, a path can actually
consist of several subpath, which may be open or not and may even
consist of only a single point (a single move-to). In this case, it is
not immediately obvious, where arrow heads should be placed. The
actual rules that \tikzname\ uses are goverened by the setting of the
key |tips|:

\begin{key}{/pgf/tips=\meta{value} (default true, initially on draw)}
  \keyalias{tikz}

  This key governs in what situations arrow tips are added to a
  path. The following \meta{values} are permissible:
  \begin{itemize}
  \item |true| (the value used when no \meta{value} is specified)
  \item |proper|
  \item |on draw| (the initial value, if the key has not yet been used
    at all)
  \item |on proper draw|
  \item |never| or |false| (same effect)
  \end{itemize}

  Firstly, there are a whole bunch of situations where the setting of
  these (or other) options causes no arrow tips to be shown:
  \begin{itemize}
  \item If no arrow tips have been specified (for instance, by having
    said |arrows=-|), no arrow tips are drawn.    
  \item If the |clip| option is set, no arrow tips are drawn.    
  \item If |tips| has been set to |never| or |false|, no arrow tips are drawn.    
  \item If |tips| has been set to |on draw| or |on proper draw|, but
    the |draw| option is not set, no arrow tips are drawn.
  \item If the path is empty (as in |\path ;|), no arrow tips are drawn.
  \item If at least one of the subpaths of a path is closed (|cycle| is
    used somewhere or something like |circle| or |rectangle|), arrow
    tips are never drawn anywhere -- even if there are open subpaths.
  \end{itemize}

  Now, if we pass all of the above tests, we must have a closer look
  at the path. All its subpaths must now be open and there must be at
  least one subpath. We conside the last one. Arrow tips will only be
  added to this last subpath.

  \begin{enumerate}
  \item If this last subpath not degenerate (all coordinates on the
    subpath are the same as in a single ``move-to'' |\path (0,0);| or
    in a ``move-to'' followed by a ``line-to'' to the same position as
    in |\path (1,2) -- (1,2)|), arrow tips are added to this last
    subpath now.
  \item If the last subpath is degenerate, we add arrow tips pointing
    upward at the single coordinate mentioned in the path, but only
    for |tips| begin set to |true| or to |on draw| -- and not for
    |proper| nor for |on proper draw|. In other words, ``proper''
    suppresses arrow tips on degenerate paths.
  \end{enumerate}

\begin{codeexample}[]
% No path, no arrow tips:
\tikz [<->] \draw; 
\end{codeexample}
\begin{codeexample}[]
% Degenerate path, draw arrow tips (but no path, it is degenerate...)
\tikz [<->] \draw (0,0); 
\end{codeexample}
\begin{codeexample}[]
% Degenerate path, tips=proper suppresses arrows
\tikz [<->] \draw [tips=proper] (0,0); 
\end{codeexample}
\begin{codeexample}[]
% Normal case:
\tikz [<->] \draw (0,0) -- (1,0); 
\end{codeexample}
\begin{codeexample}[]
% Two subpaths, only second gets tips
\tikz [<->] \draw (0,0) -- (1,0) (2,0) -- (3,0);
\end{codeexample}
\begin{codeexample}[]
% Two subpaths, second degenerate, but still gets tips
\tikz [<->] \draw (0,0) -- (1,0) (2,0);
\end{codeexample}
\begin{codeexample}[]
% Two subpaths, second degenerate, proper suppresses them
\tikz [<->] \draw [tips=on proper draw] (0,0) -- (1,0) (2,0);
\end{codeexample}
\begin{codeexample}[]
% Two subpaths, but one is closed: No tips, even though last subpath is open
\tikz [<->] \draw (0,0) circle[radius=2pt] (2,0) -- (3,0);
\end{codeexample}
\end{key}

One common pitfall when arrow tips are added to a path should be
addressed right here at the beginning: When \tikzname\ positions an
arrow tip at the start, for all its computations it only takes into
account the first segment of the subpath to which the arrow tip is
added. This ``first segment'' is the first line-to or curve-to operation (or arc
or parabola or a similar operation) of the path; but note that
decorations like |snake| will add many small line segments to
paths. The important point 
is that if this first segment is very small, namely smaller that the
arrow tip itself, strange things may result. As will be explained in
Section~\ref{section-arrow-flex}, \tikzname\ will modify the path by
shortening the first segment and shortening a segment below its length
may result in strange effects. Similarly, for tips at the end of a
subpath, only the last segment is considered.

The bottom line is that wherever an arrow tip is added to a path, the
line segment where it is added should be ``long enough.'' 




\subsection{Arrow Keys: Configuring the Appearance of a Single Arrow Tip}
\label{section-arrow-config}

For standard arrow tip kinds, like |Stealth| or |Latex| or |Bar|, 
you can easily change their size, aspect ratio, color, and other
parameters. This is similar to selecting a font face from a font
family: \emph{``This text''} is not just typeset in the font 
``Computer Modern,'' but rather in ``Computer Modern, italic face,
11pt size, medium weight, black color, no underline, \dots''
Similarly, an arrow tip is not just a ``Stealth'' arrow tip, but
rather a ``Stealth arrow tip at its natural size, flexing, but not
bending along the path, miter line caps, draw and fill colors
identical to the path draw color, \dots''

Just as most programs make it easy to ``configure'' which font should
be used at a certain point in a text, \tikzname\ tries to make it easy
to specify which configuration of an arrow tip should be used. You use
\emph{arrow keys}, where a certain parameter like the |length| of an
arrow is set to a given value using the standard key--value
syntax. You can provide several arrow keys following an arrow tip kind
in  an arrow tip specification as in
|Stealth[length=4pt,width=2pt]|.

While selecting a font may be easy, \emph{designing} a new font is a
highly creative and difficult process and more often than not, not all
faces of a font are available on any given system. The difficulties
involved in designing a new arrow tip are somewhat similar to designing a new
letter for a font and, thus, it may also happen that not all
configuration options are actually implemented for a given arrow
tip. Naturally, for the standard arrow tips, all configuration options
are available -- but for special-purpose arrow tips it may well happen
that an arrow tip kind simply ``ignores'' some of the configurations
given by you.

Some of the keys explained in the following are defined in the library
|arrows.meta|, others are always available. This has to do with the
question of whether the arrow key needs to be supported directly in the
\pgfname\ core or not. In general, the following explanations assume
that |arrows.meta| has been loaded.


\subsubsection{Size}

The most important configuration parameter of an arrow tip is
undoubtedly its size. The following two keys are the main keys that
are important in this context:

\begin{key}{/pgf/arrow keys/length=\meta{dimension}| |\opt{\meta{line width factor}}%
    | |\opt{\meta{outer factor}}}
  \label{length-arrow-key}%
  This parameter is usually the most important parameter that governs
  the size of an arrow tip: The \meta{dimension} that you provide
  dictates the distance from the ``very tip'' of the arrow to its
  ``back end'' along the line:
\begin{codeexample}[]
\tikz{
  \draw [-{Stealth[length=5mm]}] (0,0) -- (2,0);
  \draw [|<->|] (1.5,.4) -- node[above=1mm] {5mm} (2,.4);
}
\end{codeexample}
\begin{codeexample}[]
\tikz{
  \draw [-{Latex[length=5mm]}] (0,0) -- (2,0);
  \draw [|<->|] (1.5,.4) -- node[above=1mm] {5mm} (2,.4);
}
\end{codeexample}
\begin{codeexample}[]
\tikz{
  \draw [-{Classical TikZ Rightarrow[length=5mm]}] (0,0) -- (2,0);
  \draw [|<->|] (1.5,.6) -- node[above=1mm] {5mm} (2,.6);
}
\end{codeexample}

  \medskip
  \noindent \textbf{The Line Width Factors.}
  Following the \meta{dimension}, you may put a space followed by a
  \meta{line width factor}, which must be a plain number (no |pt| or
  |cm| following). When you provide such a number, the size of the
  arrow tip is not just \meta{dimension}, but rather $\meta{dimension}
  + \meta{line width factor}\cdot w$ where
  $w$ is the width of the to-be-drawn path. This
  makes it easy to vary the size of an arrow tip in accordance with
  the line width -- usually a very good idea since thicker lines will
  need thicker arrow tips. 

  As an example, when you write |length=0pt 5|, the length of the
  arrow will be exactly five times the current line width. As another
  example, the default length of a |Latex| arrow is
  |length=3pt 4.5 0.8|. Let us ignore the 0.8 for a moment; the
  |4pt 4.5| then means that for the standard line width of
  |0.4pt|, the length of a |Latex| arrow will be exactly 4.8pt (3pt
  plus 4.5 times |0.4pt|). 

  Following the line width factor, you can additionally provide an
  \meta{outer factor}, again preceded by a space (the |0.8| in the
  above example). This factor is
  taken into consideration only when the |double| option is used, that
  is, when a so-called ``inner line width''. For a double line, we can
  identify three different ``line widths'', namely the inner line
  width $w_i$, the line width  $w_o$ of the two outer lines, and the
  ``total line width'' $w_t = w_i + 2w_o$. In the below examples, we
  have $w_i = 3\mathrm{pt}$, $w_o=1\mathrm{pt}$, and $w_t =
  5\mathrm{pt}$. It is not immediately clear 
  which of these line widths should be considered as $w$ in the above
  formula $\meta{dimension} + \meta{line width factor}\cdot w$ for the
  computation of the length. One can argue both for $w_t$ and also for
  $w_o$. Because of this, you use the \meta{outer factor} to
  decide on one of them or even 
  mix them: \tikzname\ sets $w = \meta{outer factor} w_o +
  (1-\meta{outer factor})w_t$. Thus, when the outer factor is $0$, as
  in the first of the following examples and as is the default when it
  is not specified, the computed $w$ will be the total
  line width $w_t = 5\mathrm{pt}$. Since
  $w=5\mathrm{pt}$, we get a total length of $15pt$ in the first
  example (because of the factor |3|). In contrast, in the last
  example, the outer factor is 1 and, thus, $w = w_o = \mathrm{1pt}$ and the
  resulting length is 3pt. Finally, for the middle case, the ``middle'' between 5pt and
  1pt is 3pt, so the length is 9pt.
\begin{codeexample}[]
\tikz \draw [line width=1pt, double distance=3pt,
             arrows = {-Latex[length=0pt 3 0]}] (0,0) -- (1,0);
\end{codeexample}
\begin{codeexample}[]
\tikz \draw [line width=1pt, double distance=3pt,
             arrows = {-Latex[length=0pt 3 .5]}] (0,0) -- (1,0);
\end{codeexample}
\begin{codeexample}[]
\tikz \draw [line width=1pt, double distance=3pt,
             arrows = {-Latex[length=0pt 3 1]} ] (0,0) -- (1,0);
\end{codeexample}

  \medskip
  \noindent \textbf{The Exact Length.}
  For an arrow tip kind that is just an outline that is filled with a
  color, the specified length should \emph{exactly} equal the distance
  from the tip to the back end. However, when the arrow tip is drawn
  by stroking a line, it is no longer obvious whether the |length|
  should refer to the extend of the stroked lines' path or of the
  resulting pixels (which will be wider because of the thickness of
  the stroking pen). The rules are as follows:
  \begin{enumerate}
  \item If the arrow tip consists of a closed path (like |Stealth| or
    |Latex|), imagine the arrow tip drawn from left to right using a
    miter line cap. Then the |length| should be the horizontal
    distance from the first drawn ``pixel'' to the last drawn
    ``pixel''. Thus, the thickness of the stroked line and also the
    miter ends should be taken into account:
\begin{codeexample}[]
\tikz{
  \draw [line width=1mm, -{Stealth[length=10mm, open]}]
          (0,0) -- (2,0);
  \draw [|<->|] (2,.6) -- node[above=1mm] {10mm} ++(-10mm,0);
}
\end{codeexample}
  \item If, in the above case, the arrow is drawn using a round line
    join (see Section~\ref{section-arrow-key-caps} for details on how
    to select this), the size of the arrow should still be the same as
    in the first case (that is, as if a miter join were used). This
    creates some ``visual consistency'' if the two modes are mixed or
    if you later one change the mode.
\begin{codeexample}[]
\tikz{
  \draw [line width=1mm, -{Stealth[length=10mm, open, round]}]
          (0,0) -- (2,0);
  \draw [|<->|] (2,.6) -- node[above=1mm] {10mm} ++(-10mm,0);
}
\end{codeexample}
    As the above example shows, however, a rounded arrow will still
    exactly ``tip'' the point where the line should end (the point
    |(2,0)| in the above case). It is only the scaling of the arrow
    that is not affected.
  \end{enumerate}
\end{key}

\begin{key}{/pgf/arrow keys/width=\meta{dimension}| |\opt{\meta{line width factor}}%
    | |\opt{\meta{outer factor}}}
  This key works line the |length| key, only it specifies the
  ``width'' of the arrow tip; so if width and length are identical, the
  arrow will just touch the borders of a square. (An exception to this
  rule are ``halved'' arrow tips, see
  Section~\ref{section-arrow-key-harpoon}.) The meaning of the two
  optional factor numbers following the \meta{dimension} is the same
  as for the |length| key.
\begin{codeexample}[]
\tikz \draw [arrows = {-Latex[width=10pt, length=10pt]}] (0,0) -- (1,0);
\end{codeexample}
\begin{codeexample}[]
\tikz \draw [arrows = {-Latex[width=0pt 10, length=10pt]}] (0,0) -- (1,0);
\end{codeexample}
\end{key}

\begin{key}{/pgf/arrow keys/width'=\meta{dimension}| |\opt{\meta{length
        factor}| |\opt{\meta{line width factor}}}}
  The key (note the prime) has a similar effect as the |width|
  key. The difference is that the second, still optional paraemter
  \meta{length factor} specifies the width of the key not as a
  multiple of the line width, but as a multiple of the arrow length.

  The idea is that if you write, say, |width'=0pt 0.5|, the width of
  the arrow will be half its length. Indeed, for standard arrow tips
  like |Stealth| the default width is specified in this way so that if
  you change the length of an arrow tip, you also change the width in
  such a way that the aspect ratio of the arrow tip is kept. The other
  way round, if you modify the factor in |width'| without changing the
  length, you change the aspect ratio of the arrow tip.

  Note that later changes of the length are taken into account for the
  computation. For instance, if you write
\begin{codeexample}[code only]
length = 10pt, width'=5pt 2, length=7pt    
\end{codeexample}
  the resulting width will be $19\mathrm{pt} = 5\mathrm{pt} + 2\cdot
  7\mathrm{pt}$.

\begin{codeexample}[]
\tikz \draw [arrows = {-Latex[width'=0pt .5, length=10pt]}] (0,0) -- (1,0);
\end{codeexample}
\begin{codeexample}[]
\tikz \draw [arrows = {-Latex[width'=0pt .5, length=15pt]}] (0,0) -- (1,0);
\end{codeexample}
  The third, also optional, parameter allows you to add a multiple of
  the line width to the value computed in terms of the length.
\end{key}


\begin{key}{/pgf/arrow keys/inset=\meta{dimension}| |\opt{\meta{line width factor}}%
    | |\opt{\meta{outer factor}}}
  The key is relevant only for some arrow tips such as the |Stealth|
  arrow tip. It specifies a distance by which something inside the
  arrow tip is set inwards; for the |Stealth| arrow tip it is the
  distance by which the back angle is moved inwards.

  The computation of the distance works in the same way as for
  |length| and |width|: To the \meta{dimension} we add \meta{line
    width factor} times that line width, where the line width is
  computed based on the \meta{outer factor} as described for the
  |length| key.
\begin{codeexample}[]
\tikz \draw [arrows = {-Stealth[length=10pt, inset=5pt]}] (0,0) -- (1,0);
\end{codeexample}
\begin{codeexample}[]
\tikz \draw [arrows = {-Stealth[length=10pt, inset=2pt]}] (0,0) -- (1,0);
\end{codeexample}

  For most arrows for which there is no ``natural inset'' like, say,
  |Latex|, this key has no effect.
\end{key}


\begin{key}{/pgf/arrow keys/inset'=\meta{dimension}| |\opt{\meta{length factor}}| |\opt{\meta{line width factor}}}
  This key works like |inset|, only like |width'| the second parameter
  is a factor of the arrow length rather than of the line width. For
  instance, the |Stealth| arrow sets |inset'| to |0pt 0.325| to ensure
  that the inset is always at $13/40$th of the arrow length if nothing
  else is specified.
\end{key}



\begin{key}{/pgf/arrow keys/angle=\meta{angle}|:|\meta{dimension}%
    | |\opt{\meta{line width factor}}%
    | |\opt{\meta{outer factor}}}
  This key sets the |length| and the |width| of an arrow tip at the
  same time. The lenght will be the cosine of \meta{angle}, while the
  width will be twice the sine of half the \meta{angle} (this slightly
  awkward rule ensures that a |Stealth| arrow will have an opening
  angle of \meta{angle} at its tip if this option is used). As for the
  |length| key, if the optional factors are given, they add a certain
  multiple of the line width to the \meta{dimension} before the sine
  and cosines are computed.
\begin{codeexample}[]
\tikz \draw [arrows = {-Stealth[inset=0pt, angle=90:10pt]}] (0,0) -- (1,0);
\end{codeexample}
\begin{codeexample}[]
\tikz \draw [arrows = {-Stealth[inset=0pt, angle=30:10pt]}] (0,0) -- (1,0);
\end{codeexample}
\end{key}


\begin{key}{/pgf/arrow keys/angle'=\meta{angle}}
  Sets the width of the arrow to twice the tangent of $\meta{angle}/2$
  times the arrow length. This results in an arrow tip with an opening
  angle of \meta{angle} at its tip and with the specified |length|
  unchanged. 
\begin{codeexample}[]
\tikz \draw [arrows = {-Stealth[inset=0pt, length=10pt, angle'=90]}]
            (0,0) -- (1,0);
\end{codeexample}
\begin{codeexample}[]
\tikz \draw [arrows = {-Stealth[inset=0pt, length=10pt, angle'=30]}]
            (0,0) -- (1,0);
\end{codeexample}
\end{key}


\subsubsection{Scaling}

In the previous section we saw that there are many options for getting
``fine control'' overt the length and width of arrow tips. However, in
some cases, you do not really care whether the arrow tip is 4pt long
or 4.2pt long, you ``just want it to be a little bit larger than
usual.'' In such cases, the following keys are useful:

\begin{key}{/pgf/arrows keys/scale=\meta{factor} (initially 1)}
  After all the other options listed in the previous (and also the
  following sections) have been processed, \tikzname\ applies a
  \emph{scaling} to the computed length, inset, and width of the arrow
  tip (and, possibly, to other size parameters defined by
  special-purpose arrow tip kinds). Everything is simply scaled by the
  given \meta{factor}.
\begin{codeexample}[]
\tikz {
  \draw [arrows = {-Stealth[]}]          (0,1)   -- (1,1);
  \draw [arrows = {-Stealth[scale=1.5]}] (0,0.5) -- (1,0.5);
  \draw [arrows = {-Stealth[scale=2]}]   (0,0)   -- (1,0);
}
\end{codeexample}  
  Note that scaling has \emph{no} effect on the line width (as usual)
  and also not on the arrow padding (the |sep|).
\end{key}

You can get even more fine-grained control over scaling using the
following keys (the |scale| key is just a shorthand for setting both
of the following keys simultaneously):

\begin{key}{/pgf/arrows keys/scale length=\meta{factor} (initially 1)}
  This factor works like |scale|, only it is applied only to
  dimensions ``along the axis of the arrow,'' that is, to the length
  and to the inset, but not to the width. 
\begin{codeexample}[]
\tikz {
  \draw [arrows = {-Stealth[]}]                 (0,1)   -- (1,1);
  \draw [arrows = {-Stealth[scale length=1.5]}] (0,0.5) -- (1,0.5);
  \draw [arrows = {-Stealth[scale length=2]}]   (0,0)   -- (1,0);
}
\end{codeexample}  
\end{key}

\begin{key}{/pgf/arrows keys/scale width=\meta{factor} (initially 1)}
  Like |scale length|, but for dimensions related to the width.
\begin{codeexample}[]
\tikz {
  \draw [arrows = {-Stealth[]}]                 (0,1)   -- (1,1);
  \draw [arrows = {-Stealth[scale width=1.5]}] (0,0.5) -- (1,0.5);
  \draw [arrows = {-Stealth[scale width=2]}]   (0,0)   -- (1,0);
}
\end{codeexample}  
\end{key}


\subsubsection{Arc Angles}

A few arrow tips consist mainly of arcs, whose length can be
specified. For these arrow tips, you use the following key:

\begin{key}{/pgf/arrow keys/arc=\meta{degrees} (initially 180)}
  Sets the angle of arcs in arrows to \meta{degrees}. Note that this key
  is quite different from the |angle| key, which is ``just a
  fancy way of setting the length and width.'' In contrast, the |arc|
  key is used to set the degrees of arcs that are part of an arrow
  tip:
\begin{codeexample}[]
\tikz [ultra thick] {
  \draw [arrows = {-Hooks[]}]         (0,1)   -- (1,1);
  \draw [arrows = {-Hooks[arc=90]}]   (0,0.5) -- (1,0.5);
  \draw [arrows = {-Hooks[arc=270]}]  (0,0)   -- (1,0);
}
\end{codeexample}  
\end{key}


\subsubsection{Slanting}

You can ``slant'' arrow tips using the following key:

\begin{key}{/pgf/arrow keys/slant=\meta{factor} (initially 0)}
  Slanting is used to create an ``italics'' effect for arrow tips: All
  arrow tips get ``slanted'' a little bit relative to the axis of the
  arrow:
\begin{codeexample}[]
\tikz {
  \draw [arrows = {->[]}]         (0,1)   -- (1,1);
  \draw [arrows = {->[slant=.5]}] (0,0.5) -- (1,0.5);
  \draw [arrows = {->[slant=1]}]  (0,0)   -- (1,0);
}
\end{codeexample}  
  There is one thing to note about slanting: Slanting is done using a
  so-called ``canvas transformation''  and has no effect on
  positioning of the arrow tip. In particular, if an arrow tip gets
  slanted so strongly that it starts to protrude over the arrow tip
  end, this does not change the positioning of the arrow tip.

  Here is another example where slanting is used to match italic text:
\begin{codeexample}[]
\tikz [>={[slant=.3] To[] To[]}]
  \graph [math nodes] { A -> B <-> C };  
\end{codeexample}
\end{key}



\subsubsection{Reversing, Halving, Swapping}
\label{section-arrow-key-harpoon}

\begin{key}{/pgf/arrow keys/reverse}
  Adding this key to an arrow tip will ``reverse its direction'' so
  that is points in the opposite direction (but is still at that end of the
  line where the non-reversed arrow tip would have been drawn; so only
  the tip is reversed). For most arrow tips, this just results in an
  internal flip of a coordinate system, but some arrow tips actually
  use a slightly different version of the tip for reversed arrow tips
  (namely when the joining of the tip with the line would look
  strange). All of this happens automatically, so you do not need to
  worry about this.

  If you apply this key twice, the effect cancels, which is useful for
  the definition of shorthands (which will be discussed later).
\begin{codeexample}[width=3cm]
\tikz [ultra thick] \draw [arrows = {-Stealth[reversed]}] (0,0) -- (1,0);
\end{codeexample}
\begin{codeexample}[width=3cm]
\tikz [ultra thick] \draw [arrows = {-Stealth[reversed, reversed]}] (0,0) -- (1,0);
\end{codeexample}
\end{key}

\begin{key}{/pgf/arrow keys/harpoon}
  The key requests that only the ``left half'' of the arrow tip should
  drawn:
\begin{codeexample}[width=3cm]
\tikz [ultra thick] \draw [arrows = {-Stealth[harpoon]}] (0,0) -- (1,0);
\end{codeexample}
\begin{codeexample}[width=3cm]
\tikz [ultra thick] \draw [arrows = {->[harpoon]}] (0,0) -- (1,0);
\end{codeexample}
  Unlike the |reverse| key, which all arrows tip kinds support at
  least in a basic way, designers of arrow tips really need to take
  this key into account in their arrow tip code and often a lot of
  special attention needs to do be paid to this key in the
  implementation. For this reason, only some arrow tips will support
  it. 
\end{key}

\begin{key}{/pgf/arrow keys/swap}
  This key flips that arrow tip along the axis of the line. It makes
  sense only for asymmetric arrow tips like the harpoons created using
  the |harpoon| option. 
\begin{codeexample}[width=3cm]
\tikz [ultra thick] \draw [arrows = {-Stealth[harpoon]}] (0,0) -- (1,0);
\end{codeexample}
\begin{codeexample}[width=3cm]
\tikz [ultra thick] \draw [arrows = {-Stealth[harpoon,swap]}] (0,0) -- (1,0);
\end{codeexample}
  Swapping is always possible, no special code is needed on behalf of
  an arrow tip implementer. 
\end{key}

\begin{key}{/pgf/arrow keys/left}
  A shorthand for |harpoon|.  
\end{key}

\begin{key}{/pgf/arrow keys/right}
  A shorthand for |harpoon, swap|.
\begin{codeexample}[width=3cm]
\tikz [ultra thick] \draw [arrows = {-Stealth[left]}] (0,0) -- (1,0);
\end{codeexample}
\begin{codeexample}[width=3cm]
\tikz [ultra thick] \draw [arrows = {-Stealth[right]}] (0,0) -- (1,0);
\end{codeexample}
\end{key}


\subsubsection{Coloring}

Arrow tips are drawn using the same basic mechanisms as normal paths,
so arrow tips can be stroked (drawn) and/or filled. However, we
usually want the color of arrow tips to be identical to the color used
to draw the path, even if a different color is used for filling the
path. On the other hand, we may also sometimes wish to use a special
color for the arrow tips that is different from both the line and fill
colors of the main path.

The following options allow you to configure how arrow tips are
colored:

\begin{key}{/pgf/arrow keys/color=\meta{color or empty} (initially
    \normalfont empty)}
  Normally, an arrow tip gets the same color as the path to which it
  is attached. More precisely, it will get the current ``draw color'',
  also known as ``stroke color,'' which you can set using
  |draw=|\meta{some color}. By adding the option |color=| to an arrow
  tip (note that an ``empty'' color is specified in this way), you ask
  that the arrow tip gets this default draw color of the path. Since
  this is the default behaviour, you usually do not need to specify
  anything: 
\begin{codeexample}[width=3cm]
\tikz [ultra thick] \draw [red, arrows = {-Stealth}] (0,0) -- (1,0);
\end{codeexample}
\begin{codeexample}[width=3cm]
\tikz [ultra thick] \draw [blue, arrows = {-Stealth}] (0,0) -- (1,0);
\end{codeexample}

  Now, when you provide a \meta{color} with this option, you request
  that the arrow tip should get this color \emph{instead} of the color
  of the main path:
\begin{codeexample}[width=3cm]
\tikz [ultra thick] \draw [red, arrows = {-Stealth[color=blue]}] (0,0) -- (1,0);
\end{codeexample}
\begin{codeexample}[width=3cm]
\tikz [ultra thick] \draw [red, arrows = {-Stealth[color=black]}] (0,0) -- (1,0);
\end{codeexample}

  Similar to the |color| option used in normal \tikzname\ options, you
  may omit the |color=| part of the option. Whenever an \meta{arrow key}
  is  encountered that \tikzname\ does not recognize, it will test whether
  the key is the name of a color and, if so, execute
  |color=|\meta{arrow key}. So, the first of the above examples can be
  rewritten as follows:
\begin{codeexample}[width=3cm]
\tikz [ultra thick] \draw [red, arrows = {-Stealth[blue]}] (0,0) -- (1,0);
\end{codeexample}

  The \meta{color} will apply both to any drawing and filling
  operations used to construct the path. For instance, even though the
  |Stealth| arrow tips looks like a filled quadrilateral, it is
  actually constructed by drawing a quadrilateral and then filling it
  in the same color as the drawing (see the |fill| option below to
  see the difference).

  When |color| is set to an empty text, the drawing color is
  always used to fill the arrow tips, even if a different color is
  specified for filling the path: 
\begin{codeexample}[width=3cm]
\tikz [ultra thick] \draw [draw=red, fill=red!50, arrows = {-Stealth[length=10pt]}]
                          (0,0) -- (1,1) -- (2,0);
\end{codeexample}
  As you can see in the above example, the filled area is not quite
  what you might have expected. The reason is that the path was
  actually internally shortened a bit so that the end of the ``fat
  line'' as inside the arrow tip and we get a ``clear'' arrow tip.

  In general, it is a good idea not to add arrow tips to paths that
  are filled.
\end{key}

\begin{key}{/pgf/arrow keys/fill=\meta{color or |none|}}
  Use this key to explicitly set the color used for filling the arrow
  tips. This color can be different from the color used to draw
  (stroke) the arrow tip:
\begin{codeexample}[width=3cm]
\tikz {
  \draw [help lines] (0,-.5) grid [step=1mm] (1,.5);
  \draw [thick, red, arrows = {-Stealth[fill=white,length=15pt]}] (0,0) -- (1,0);
}
\end{codeexample}
  You can also specify the special ``color'' |none|. In this case, the
  arrow tip is not filled at all (not even with white):
\begin{codeexample}[width=3cm]
\tikz {
  \draw [help lines] (0,-.5) grid [step=1mm] (1,.5);
  \draw [thick, red, arrows = {-Stealth[fill=none,length=15pt]}] (0,0) -- (1,0);
}
\end{codeexample}
  Note that such ``open'' arrow tips are a bit difficult to draw in
  some case: The problem is that the line must be shortened by just
  the right amount so that it ends exactly on the back end of the
  arrow tip. In some cases, especially when double lines are used,
  this will not be possible.

  \begin{key}{/pgf/arrow keys/open}
    A shorthand for |fill=none|.  
  \end{key}
  
  When you use both the |color| and |fill| option, the |color| option
  must come first since it will reset the filling to the color
  specified for drawing.
\begin{codeexample}[width=3cm]
\tikz {
  \draw [help lines] (0,-.5) grid [step=1mm] (1,.5);
  \draw [thick, red, arrows = {-Stealth[color=blue, fill=white, length=15pt]}] 
        (0,0) -- (1,0);
}
\end{codeexample}

  Note that by setting |fill| to the special color |pgffillcolor|, you
  can cause the arrow tips to be filled using the color used to fill
  the main path. (This special color is always available and always
  set to the current filling color of the graphic state.):
\begin{codeexample}[width=3cm]
\tikz [ultra thick] \draw [draw=red, fill=red!50,
                           arrows = {-Stealth[length=15pt, fill=pgffillcolor]}]
                          (0,0) -- (1,1) -- (2,0);
\end{codeexample}
\end{key}


\subsubsection{Line Styling}
\label{section-arrow-key-caps}

Arrow tips are created by drawing and possibly filling a path that
makes up the arrow tip. When \tikzname\ draws a path, there are
different ways in which such a path can be drawn (such as
dashing). Three particularly important parameters are the line join,
the line cap, see Section~\ref{section-line-cap} for an introduction,
and the line width (thickness) 

\tikzname\ resets the line cap and line join each time it draws an
arrow tip since you usually do not want their settings to ``spill
over'' to the way the arrow tips are drawn. You can, however, change
there values explicitly for an arrow tip:

\begin{key}{/pgf/arrow keys/line cap=\meta{|round| or |butt|}}
  Sets the line cap of all lines that are drawn in the arrow to a
  round cap or a butt cap. (Unlike for normal lines, the |rect| cap is
  not allowed.) Naturally, this key has no effect for arrows whose
  paths are closed.

  Each arrow tip has a default value for the line cap, which can
  be overruled using this option.

  Changing the cap should have no effect on the size of the
  arrow. However, it will have an effect on where the exact ``tip'' of
  the arrow is since this will always be exactly at the end of the
  arrow:
\begin{codeexample}[width=3cm]
\tikz [line width=2mm]
  \draw [arrows = {-Computer Modern Rightarrow[line cap=butt]}]
        (0,0) -- (1,0);
\end{codeexample}
\begin{codeexample}[width=3cm]
\tikz [line width=2mm]
  \draw [arrows = {-Computer Modern Rightarrow[line cap=round]}]
        (0,0) -- (1,0);
\end{codeexample}
\begin{codeexample}[width=3cm]
\tikz [line width=2mm]
  \draw [arrows = {-Bracket[reversed,line cap=butt]}]
        (0,0) -- (1,0);
\end{codeexample}
\begin{codeexample}[width=3cm]
\tikz [line width=2mm]
  \draw [arrows = {-Bracket[reversed,line cap=round]}]
        (0,0) -- (1,0);
\end{codeexample}
\end{key}

\begin{key}{/pgf/arrow keys/line join=\meta{|round| or |miter|}}
  Sets the line join to round or miter (|bevel| is not allowed). This
  time, the key only has an effect on paths that have ``corners'' in
  them. The same rules as for |line cap| apply: the size is not
  affects, but the tip end is:
 \begin{codeexample}[width=3cm]
\tikz [line width=2mm]
  \draw [arrows = {-Computer Modern Rightarrow[line join=miter]}]
        (0,0) -- (1,0);
\end{codeexample}
\begin{codeexample}[width=3cm]
\tikz [line width=2mm]
  \draw [arrows = {-Computer Modern Rightarrow[line join=round]}]
        (0,0) -- (1,0);
\end{codeexample}
\begin{codeexample}[width=3cm]
\tikz [line width=2mm]
  \draw [arrows = {-Bracket[reversed,line join=miter]}]
        (0,0) -- (1,0);
\end{codeexample}
\begin{codeexample}[width=3cm]
\tikz [line width=2mm]
  \draw [arrows = {-Bracket[reversed,line join=round]}]
        (0,0) -- (1,0);
\end{codeexample}
\end{key} 

The following keys set both of the above:
\begin{key}{/pgf/arrow keys/round}
  A shorthand for |line cap=round, line join=round|, resulting in
  ``rounded'' arrow heads.  
 \begin{codeexample}[width=3cm]
\tikz [line width=2mm]
  \draw [arrows = {-Computer Modern Rightarrow[round]}] (0,0) -- (1,0);
\end{codeexample}
\begin{codeexample}[width=3cm]
\tikz [line width=2mm]
  \draw [arrows = {-Bracket[reversed,round]}] (0,0) -- (1,0);
\end{codeexample}
\end{key}

\begin{key}{/pgf/arrow keys/sharp}
  A shorthand for |line cap=butt, line join=miter|, resulting in
  ``sharp'' or ``pointed'' arrow heads.  
 \begin{codeexample}[width=3cm]
\tikz [line width=2mm]
  \draw [arrows = {-Computer Modern Rightarrow[sharp]}] (0,0) -- (1,0);
\end{codeexample}
\begin{codeexample}[width=3cm]
\tikz [line width=2mm]
  \draw [arrows = {-Bracket[reversed,sharp]}] (0,0) -- (1,0);
\end{codeexample}
\end{key}


You can also set the width of lines used inside arrow tips:

\begin{key}{/pgf/arrow keys/line width=\meta{dimension}| |\opt{\meta{line width factor}}%
    | |\opt{\meta{outer factor}}}
  This key sets the line width inside an arrow tip for drawing
  (out)lines of the arrow tip. When you set this width to |0pt|, which
  makes sense only for closed tips, the arrow tip is only filled. This
  can result in better rendering of some small arrow tips and in case
  of bend arrow tips (because the line joins will also be bend and not
  ``mitered''.)

  The meaning of the factors is as usual the same as for |length| or |width|.
  
\begin{codeexample}[width=2cm]
\tikz \draw [arrows = {-Latex[line width=0.1pt, fill=white, length=10pt]}] (0,0) -- (1,0);
\end{codeexample}
\begin{codeexample}[width=2cm]
\tikz \draw [arrows = {-Latex[line width=1pt, fill=white, length=10pt]}] (0,0) -- (1,0);
\end{codeexample}
\end{key}


\begin{key}{/pgf/arrow keys/line width'=\meta{dimension}| |\opt{\meta{length factor}}}
  Works like |line width| only the factor is with respect to the |length|.  
\end{key}

\subsubsection{Bending and Flexing}

\label{section-arrow-flex}

Up to now, we have only added arrow tip to the end of straight lines,
which is in some sense ``easy.'' Things get far more difficult, if the
line to which we wish to end an arrow tip is curved. In the following,
we have a look at the different actions that can be taken and how they
can be configured.

To get a feeling for the difficulties involved, consider the following
situation: We have a ``gray wall'' at the $x$-coordinate of and a
red line that ends in its middle.

\begin{codeexample}[]
\def\wall{ \fill     [fill=black!50]  (1,-.5) rectangle (2,.5);
           \pattern  [pattern=bricks] (1,-.5) rectangle (2,.5);
           \draw     [line width=1pt]  (1cm+.5pt,-.5) -- ++(0,1); }
\begin{tikzpicture}
  \wall
  % The "line"
  \draw [red,line width=1mm] (-1,0) -- (1,0);
\end{tikzpicture}
\end{codeexample}

Now we wish to add a blue open arrow tip the red line like, say, 
|Stealth[length=1cm,open,blue]|: 

\def\wall{ \fill     [fill=black!50]  (1,-.5) rectangle (2,.5);
           \pattern  [pattern=bricks] (1,-.5) rectangle (2,.5);
           \draw     [line width=1pt]  (1cm+.5pt,-.5) -- ++(0,1); }
\begin{codeexample}[]
\begin{tikzpicture}
  \wall
  \draw [red,line width=1mm,-{Stealth[length=1cm,open,blue]}]
        (-1,0) -- (1,0);
\end{tikzpicture}
\end{codeexample}

There are several noteworthy things about the blue arrow tip:
\begin{enumerate}
\item Notice that the red line no longer goes all the way to the
  wall. Indeed, the red line ends more or less exactly where it meets
  the blue line, leaving the arrow tip empty. Now, recall that the red
  line was supposed to be the path |(-2,0)--(1,0)|; however, this path
  has obviously become much shorter (by 6.25mm to be precise). This
  effect is called \emph{path shortening} in \tikzname.
\item The very tip of the arrow just ``touches'' the wall, even we
  zoom out a lot. This point, where the original path ended and where
  the arrow tip should now lie, is called the \emph{tip end} in
  \tikzname.
\item Finally, the point where the red line touches the blue line is
  the point where the original path ``visually ends.'' Notice that
  this is not the same as the point that lies at a distance of the
  arrow's |length| from the wall -- rather it lies at a distance of
  |length| minus the |inset|. Let us call this point the \emph{visual
    end} of the arrow.
\end{enumerate}

As pointed out earlier, for straight lines, shortening the path and
rotating and shifting the arrow tip so that it ends precisely at the
tip end and the visual end lies on a line from the tip end to the
start of the line is relatively easy.

For curved lines, things are much more difficult and \tikzname\ copes
with the difficulties in different ways, depending on which options
you add to arrows. Here is now a curved red line to which we wish to
add our arrow tip (the original straight red line is shown in light red):

\begin{codeexample}[]
\begin{tikzpicture}
  \wall
  \draw [red!25,line width=1mm] (-1,0) -- (1,0);
  \draw [red,line width=1mm] (-1,-.5) .. controls (0,-.5) and (0,0) .. (1,0);
\end{tikzpicture}
\end{codeexample}

The first way of dealing with curved lines is dubbed the ``quick
and dirty'' way (although the option for selecting this option is
politely just called ``|quick|'' \dots):

\begin{key}{/pgf/arrow keys/quick}
  Recall that curves in \tikzname\ are actually B\'ezier curves, which
  means that they start and end at certain points and we specify two
  vectors, one for the start and one for the end, that provide
  tangents to the curve at these points. In particular, for the end of
  the curve, there is a point called the \emph{second support point}
  of the curve such that a tangent to the curve at the end goes
  through this point. In our above example, the second support point
  is at the middle of the light red line and, indeed, a tangent to the
  red line at the point touching the wall is perfectly horizontal.
  
  In order to add our arrow tip to the curved path, our first
  objective is to ``shorten'' the path by 6.25mm. Unfortunately, this
  is now much more difficult than for a straight path. When the
  |quick| option is added to an arrow tip (it is also
  the default if no special libraries are loaded), we cheat somewhat:
  Instead of really moving along 6.25mm along the path, we simply
  shift the end of the curve by 6.25mm \emph{along the tangent} (which
  is easy to compute). We also have to shift the second support point
  by the same amount to ensure that the line still has the same
  tangent at the end. This will result in the following:

\begin{codeexample}[]
\begin{tikzpicture}
  \wall
  \draw [red!25,line width=1mm] (-1,0) -- (1,0);
  \draw [red,line width=1mm,-{Stealth[length=1cm,open,blue,quick]}]
        (-1,-.5) .. controls (0,-.5) and (0,0) .. (1,0);
\end{tikzpicture}
\end{codeexample}

  They main problem with the above picture is that the red line is no
  longer equal to the original red line (notice much sharper curvature
  near its end). In our example this is not such a bad thing, but it
  certainly ``not a nice thing'' that adding arrow tips to a curve
  changes the overall shape of the curves. This is especially
  bothersome if there are several similar curves that have different
  arrow heads. In this case, the similar curves now suddenly look
  different.

  Another big problem with the above approach is that it works only
  well if there is only a single arrow tip. When there are multiple
  ones, simply shifting them along the tangent as the |quick| option
  does produces less-than-satisfactory results:
\begin{codeexample}[]
\begin{tikzpicture}
  \wall
  \draw [red!25,line width=1mm] (-1,0) -- (1,0);
  \draw [red,line width=1mm,-{[quick,sep]>>>}]
        (-1,-.5) .. controls (0,-.5) and (0,0) .. (1,0);
\end{tikzpicture}
\end{codeexample}
  Note that the third arrow tip does not really lie on the curve any 
  more. 
\end{key}

Because of the shortcomings of the |quick| key, more powerful
mechanisms for shortening lines and rotating arrows tips have been
implemented. To use them, you need to load the following library:

\begin{tikzlibrary}{bending}
  Load this library to use the |flex|, |flex'|, or |bending| arrow
  keys. When this library is loaded, |flex| becomes the default mode
  that is used with all paths, unless |quick| is explicitly selected
  for the arrow tip.
\end{tikzlibrary}

\begin{key}{/pgf/arrow keys/flex=\opt{\meta{factor}} (default 1)}
  When the |bending| library is loaded, this key is applied 
  to all arrow tips by default. It has the following effect:
  \begin{enumerate}
  \item Instead of simply shifting the visual end of the arrow along
    the tangent of the curve's end, we really move it along the curve
    by the necessary distance. This operation is more expensive than
    the |quick| operation -- but not \emph{that} expensive, only
    expensive enough so that it is not selected by default for all
    arrow tips. Indeed, some compromises are made in the
    implementation where accuracy was traded for speed, so the
    distance by which the line end is shifted is not necessarily
    \emph{exactly} 6.25mm; only something reasonably close.
  \item The supports of the line are updated accordingly so that the
    shortened line will still follow \emph{exactly} the original
    line. This means that the curve deformation effect caused by the
    |quick| command does not happen here.
  \item Next, the arrow tip is rotated and shifted as follows: First,
    we shift it so that its tip is exactly at the tip end, where the
    original line ended. Then, the arrow is rotated so the \emph{the
      visual end lies on the line}:
\begin{codeexample}[]
\begin{tikzpicture}
  \wall
  \draw [red!25,line width=1mm] (-1,0) -- (1,0);
  \draw [red,line width=1mm,-{Stealth[length=1cm,open,blue,flex]}]
        (-1,-.5) .. controls (0,-.5) and (0,0) .. (1,0);
\end{tikzpicture}
\end{codeexample}
  \end{enumerate}

  As can be seen in the example, the |flex| option gives a result that
  is visually pleasing and does not deform the path.
  
  There is, however, one possible problem with the |flex| option: The
  arrow tip no longer points along the tangent of the end of the
  path. This may or may not be a problem, put especially for larger
  arrow tips readers will use the orientation of the arrow head to
  gauge the direction of the tangent of the line. If this tangent is
  important (for example, if it should be horizontal), then it may be
  necessary to enforce that the arrow tip ``really points in the
  direction of the tangent.

  To achieve this, the |flex| option takes an optional \meta{factor}
  parameter, which defaults to |1|. This factor specifies how much the
  arrow tip should be rotated: If set to |0|, the arrow points exactly
  along a tangent to curve at its tip. If set to |1|, the arrow point
  exactly along a line from the visual end point on the curve to the
  tip. For values in the middle, we interpolate the rotation between
  these two extremes; so |flex=.5| will rotate the arrow's visual end
  ``halfway away from the tangent towards the actual position on the
  line.'' 
\begin{codeexample}[]
\begin{tikzpicture}
  \wall
  \draw [red!25,line width=1mm] (-1,0) -- (1,0);
  \draw [red,line width=1mm,-{Stealth[length=1cm,open,blue,flex=0]}]
        (-1,-.5) .. controls (0,-.5) and (0,0) .. (1,0);
\end{tikzpicture}
\end{codeexample}
\begin{codeexample}[]
\begin{tikzpicture}
  \wall
  \draw [red!25,line width=1mm] (-1,0) -- (1,0);
  \draw [red,line width=1mm,-{Stealth[length=1cm,open,blue,flex=.5]}]
        (-1,-.5) .. controls (0,-.5) and (0,0) .. (1,0);
\end{tikzpicture}
\end{codeexample}
  Note how in the above examples the red line is visible inside the
  open arrow tip. Open arrow tips do not go well with a flex value
  other than~|1|. Here is a more realistic use of the |flex=0| key:
\begin{codeexample}[]
\begin{tikzpicture}
  \wall
  \draw [red!25,line width=1mm] (-1,0) -- (1,0);
  \draw [red,line width=1mm,-{Stealth[length=1cm,flex=0]}]
        (-1,-.5) .. controls (0,-.5) and (0,0) .. (1,0);
\end{tikzpicture}
\end{codeexample}
  If there are several arrow tips on a path, the |flex| option
  positions them independently, so that each of them lies optimally on
  the path:
\begin{codeexample}[]
\begin{tikzpicture}
  \wall
  \draw [red!25,line width=1mm] (-1,0) -- (1,0);
  \draw [red,line width=1mm,-{[flex,sep]>>>}]
        (-1,-.5) .. controls (0,-.5) and (0,0) .. (1,0);
\end{tikzpicture}
\end{codeexample}
\end{key}


\begin{key}{/pgf/arrow keys/flex'=\opt{\meta{factor}} (default 1)}
  The |flex'| key is almost identical to the |flex| key. The only
  difference is that a factor of |1| corresponds to rotating the arrow
  tip so that the instead of the visual end, the ``ultimate back end''
  of the arrow tip lies on the red path. In the example instead of
  having the arrow tip at a distance of |6.25mm| from the tip lie on
  the path, we have the point at a distance of |1cm| from the tip lie
  on the path:
\begin{codeexample}[]
\begin{tikzpicture}
  \wall
  \draw [red!25,line width=1mm] (-1,0) -- (1,0);
  \draw [red,line width=1mm,-{Stealth[length=1cm,open,blue,flex']}]
        (-1,-.5) .. controls (0,-.5) and (0,0) .. (1,0);
\end{tikzpicture}
\end{codeexample}
  Otherwise, the factor works as for |flex| and, indeed |flex=0| and
  |flex'=0| have the same effect.

  The main use of this option is not so much with an arrow tip like
  |Stealth| but rather with tips like the standard |>| in the context
  of a strongly curved line:
\begin{codeexample}[]
\begin{tikzpicture}
  \wall
  \draw [red!25,line width=1mm] (-1,0) -- (1,0);
  \draw [red,line width=1mm,-{Computer Modern Rightarrow[flex]}]
        (0,-.5) .. controls (1,-.5) and (0.5,0) .. (1,0);
\end{tikzpicture}
\end{codeexample}
  In the example, the |flex| option does not really flex the arrow
  since for a tip like the Computer Modern arrow, the visual end is
  the same as the arrow tip -- after all, the red line does, indeed, end
  almost exactly where it used to end.

  Nevertheless, you may feel that the arrow tip looks ``wrong'' in the sense that it
  should be rotated. This is exactly what the |flex'| option does
  since it allows us to align the ``back end'' of the tip with the red line:
\begin{codeexample}[]
\begin{tikzpicture}
  \wall
  \draw [red!25,line width=1mm] (-1,0) -- (1,0);
  \draw [red,line width=1mm,-{Computer Modern Rightarrow[flex'=.75]}]
        (0,-.5) .. controls (1,-.5) and (0.5,0) .. (1,0);
\end{tikzpicture}
\end{codeexample}
  In the example, I used |flex'=.75| so as not to overpronounce the
  effect. Usually, you will have to fiddle with it sometime to get the
  ``perfectly aligned arrow tip,'' but a value of |.75| is usually a
  good start.
\end{key}



\begin{key}{/pgf/arrow keys/bend}
  \emph{Bending} an arrow tip is a radical solution to the problem of
  positioning arrow tips on a curved line: The arrow tip is no longer
  ``rigid'' but the drawing itself will now bend along the curve. This
  has the advantage that all the problems of flexing with wrong
  tangents and overflexing disappear. The downsides are longer
  computation times (bending an arrow is \emph{much} more expensive
  that flexing it, let alone than quick mode) and also the fact that
  excessive bending can lead to ugly arrow tips. On the other hand,
  for most arrow tips their bend version are visually quite pleasing
  and create a sophisticated look:
\begin{codeexample}[]
\begin{tikzpicture}
  \wall
  \draw [red!25,line width=1mm] (-1,0) -- (1,0);
  \draw [red,line width=1mm,-{Stealth[length=20pt,bend]}]
        (-1,-.5) .. controls (0,-.5) and (0,0) .. (1,0);
\end{tikzpicture}
\end{codeexample}  
\begin{codeexample}[]
\begin{tikzpicture}
  \wall
  \draw [red!25,line width=1mm] (-1,0) -- (1,0);
  \draw [red,line width=1mm,-{[bend,sep]>>>}]
        (-1,-.5) .. controls (0,-.5) and (0,0) .. (1,0);
\end{tikzpicture}
\end{codeexample}  
\begin{codeexample}[]
\begin{tikzpicture}
  \wall
  \draw [red!25,line width=1mm] (-1,0) -- (1,0);
  \draw [red,line width=1mm,-{Stealth[bend,round,length=20pt]}]
        (0,-.5) .. controls (1,-.5) and (0.25,0) .. (1,0);
\end{tikzpicture}
\end{codeexample}  
\end{key}






\subsection{Arrow Tip Specifications}
\label{section-arrow-spec}

\subsubsection{Syntax}

When you select the arrow tips for the start and the end of a path,
you can specify a whole sequence of arrow tips, each having its own
local options. At the beginning of this section, it was pointed out
that the syntax for selecting the start and end arrow tips is the
following:

\begin{quote}
  \meta{start specification}|-|\meta{end specification}
\end{quote}

We now have a closer look at what these specifications may look
like. The general syntax of the \meta{start specification} is as
follows: 
\begin{quote}
  \opt{|[|\meta{options for all tips}|]|} \meta{first arrow tip spec}
  \meta{second arrow tip spec} \meta{third arrow tip spec} \dots
\end{quote}
As can be seen, an arrow tip specification may start with some options
in brackets. If this is the case, the \meta{options for all tips}
will, indeed, be applied to all arrow tips that follow. (We will see,
in a moment, that there are even more places where options may be
specified and a list of the ordering in which the options are applied
will be given later.)

The main part of a specification is taken up by a sequence of
individual arrow tip specifications. Such a specification can be of
three kinds:

\begin{enumerate}
\item It can be of the form \meta{arrow tip kind name}|[|\meta{options}|]|.
\item It can be of the form \meta{shorthand}|[|\meta{options}|]|.
\item It can be of the form \meta{single char shorthand}\opt{|[|\meta{options}|]|}. Note that only for this form the
  brackets are optional.
\end{enumerate}

The easiest kind is the first one: This adds an arrow tip of the kind
\meta{arrow tip kind name} to the sequence of arrow tips with the
\meta{options} applied to it at the start (for the \meta{start
  specification}) or at the end (for the \meta{end
  specification}). Note that for the \meta{start specification} the 
first arrow tip specified in this way will be at the very start of the
curve, while for the \meta{end specification} the ordering is
reversed: The last arrow tip specified will be at the very end of the
curve. This implies that a specification like
\begin{quote}
|Stealth[] Latex[] - Latex[] Stealth[]|
\end{quote}
will give perfectly symmetric arrow tips on a line (as one would
expect).

It is important that even if there are no \meta{options} for an arrow
tip, the square brackets still need to be written to indicate the end
of the arrow tip's name. Indeed, the opening brackets are used to
divide the arrow tip specification into names.

Instead of a \meta{arrow tip kind name}, you may also provide the name
of a so-called \emph{shorthand}. Shorthands look like normal arrow tip
kind names and, indeed, you will often be using shorthands without
noticing that you do. The idea is that instead of, say,
|Computer Modern Rightarrow| you might wish to just write |Rightarrow|
or perhaps just |To| or even just |>|. For this, you can create a
shorthand that tells \tikzname\ that whenever this shorthand is used,
another arrow tip kind is meant. (Actually, shorthands are somewhat
more powerful, we have a detailed look at them in
Section~\ref{section-arrow-tip-macro}.) For shorthands, the same rules
apply as for normal arrow tip kinds: You \emph{need} to provide
brackets so that \tikzname\ can find the end of the name inside a
longer specification.

The third kind of arrow tip specifications consist of just a single
letter like |>| or |)| or |*| or even |o| or |x| (but you may not use
|[|, |]|, or |-| since they will confuse the parser). These single
letter arrow specifications will invariably be shorthands that select
some ``real'' arrow tip instead. An important feature of single letter
arrow tips is that they do \emph{not} need to be followed by options
(but they may).

Now, since we can use any letter for single letter shorthands, how can
\tikzname\ tell whether by |foo[]| we mean an arrow tip kind |foo|
without any options or whether we mean an arrow tip called |f|,
followed by two arrow tips called |o|? Or perhaps an arrow tip called
|f| followed by an arrow tip called |oo|? To solve this problem, the
following rule is used to determine which of the three possible
specifications listed above applies: First, we check whether
everything from the current position up to the next opening bracket
(or up to the end) is the name of an arrow tip or of a shorthand. In
our case, |foo| would first be tested under this rule. Only if |foo|
is neither the name of an arrow tip kind nor of a shorthand does
\tikzname\ consider the first letter of the specification, |f| in our
case. If this is not the name of a shorthand, an error is
raised. Otherwise the arrow tip corresponding to |f| is added to the
list of arrow tips and the process restarts with the rest. Thus, we
would next text whether |oo| is the name of an arrow tip or shorthand
and, if not, whether |o| is such a name.

All of the above rules mean that you can rather easily specify arrow
tip sequences if they either mostly consist of single letter names or
of longer names. Here are some examples:

\begin{itemize}
\item |->>>| is interpreted as three times the |>| shorthand since
  |>>>| is not the name of any arrow tip kind (and neither is |>>|).
\item |->[]>>| has the same effect as the above.
\item |-[]>>>| also has the same effect.%>>
\item |->[]>[]>[]| so does this.
\item |->Stealth| yields an arrow tip |>| followed by a |Stealth|
  arrow at the end.
\item |-Stealth>| is illegal since there is no arrow tip |Stealth>|
  and since |S| is also not the name of any arrow tip.
\item |-Stealth[] >| is legal and does what was presumably meant in the previous item.
\item |< Stealth-| is legal and is the counterpart to |-Stealth[] >|.
\item |-Stealth[length=5pt] Stealth[length=6pt]| selects two stealth
  arrow tips, but at slightly different sizes for the end of lines.
\end{itemize}


An interesting question concerns how flexing and bending interact with
multiple arrow tips: After all, flexing and quick mode use different
ways of shortening the path so we cannot really mix them. The
following rule is used: We check, independently for the start and the
end specifications, whether at least one arrow tip in them uses one of
the options |flex|, |flex'|, or |bend|. If so, all |quick| settings in
the other arrow tips are ignored and treated as if |flex| had been
selected for them, too.


\subsubsection{Specifying Paddings}

When you provide several arrow tips in a row, all of them are added to
the start or end of the line:
\begin{codeexample}[]
\tikz \draw [<<<->>>>] (0,0) -- (2,0);
\end{codeexample}
The question is now what will be distance between them? For this, the
following arrow key is important:
\begin{key}{/pgf/arrow keys/sep=\meta{dimension}| |\opt{\meta{line
        width factor}}| |\opt{\meta{outer factor}} (default 0.88pt .3 1)}
  When a sequence of arrow tips is specified in an arrow tip
  specification for the end of the line, the arrow tips are normally
  arranged in such a way that the tip of each arrow ends exactly at
  the ``back end'' of the next arrow tip (for start specifications,
  the ordering is inverted, of course). Now, when the |sep| option is
  set, instead of exactly touching the back end of the next arrow, the
  specified \meta{dimension} is added as additional space (the
  distance may also be negative, resulting in an overlap of the arrow
  tips). The optional factors have the same meaning as for the
  |length| key, see that key for details.

  Let us now have a look at some examples. First, we use two arrow
  tips with different separations between them:
\begin{codeexample}[]
\tikz {
  \draw [-{>[sep=1pt]>[sep= 2pt]>}] (0,1.0) -- (1,1.0);
  \draw [-{>[sep=1pt]>[sep=-2pt]>}] (0,0.5) -- (1,0.5);
  \draw [-{>         >[sep]     >}] (0,0.0) -- (1,0.0);
}
\end{codeexample}

  You can also specify a |sep| for the last arrow tip in the sequence
  (for end specifications, otherwise for the first arrow tip). In this
  case, this first arrow tip will not exactly ``touch'' the point
  where the path ends, but will rather leave the specified amount of
  space. This is usually quite desirable.
\begin{codeexample}[]
\tikz {
  \node [draw] (A) {A};
  \node [draw] (B) [right=of A] {B};
  
  \draw [-{>>[sep=2pt]}] (A) to [bend left=45] (B);
  \draw [- >>          ] (A) to [bend right=45] (B);
}
\end{codeexample}
  Indeed, adding a |sep| to an arrow tip is \emph{very} desirable, so
  you will usually write something like |>={To[sep]}| somewhere near
  the start of your files.

  One arrow tip kind can be quite useful in this context: The arrow
  tip kind |_|. It draws nothing and has zero length, \emph{but} 
  it has |sep| set as a default option. Since it is a single letter
  shorthand, you can write short and clean ``code'' in this way:
\begin{codeexample}[]  
\tikz \draw [->_>] (0,0) -- (1,0);
\end{codeexample}
\begin{codeexample}[]  
\tikz \draw [->__>] (0,0) -- (1,0);
\end{codeexample}
  However, using the |sep| option will be faster than using the |_|
  arrow tip and it also allows you to specify the desired length
  directly. 
\end{key}



\subsubsection{Specifying the Line End}

In the previous examples of sequences of arrow tips, the line of the
path always ended at the last of the arrow tips (for end
specifications) or at the first of the arrow tips (for start
specifications). Often, this is what you may want, but not
always. Fortunately, it is quite easy to specify the desired end of
the line: The special single char shorthand |.| is reserved to
indicate that last arrow that is still part of the line; in other
words, the line will stop at the last arrow before |.| is encountered
(for end specifications) are at the first arrow following |.| (for
start specifications).
%>>
\begin{codeexample}[]
\tikz [very thick] \draw [<<<->>>] (0,0) -- (2,0);
\end{codeexample}
\begin{codeexample}[]
\tikz [very thick] \draw [<.<<->.>>] (0,0) -- (2,0);
\end{codeexample}
\begin{codeexample}[]
\tikz [very thick] \draw [<<.<-.>>>] (0,0) -- (2,0);
\end{codeexample}
\begin{codeexample}[]
\tikz [very thick] \draw [<<.<->.>>] (0,0) to [bend left] (2,0);
\end{codeexample}

It is permissible that there are several dots in a specification, in
this case the first one ``wins'' (for end specifications, otherwise
the last one).

Note that |.| is parsed as any other shorthand. In particular, if you
wish to add a dot after a normal arrow tip kind name, you need to add
brackets:
\begin{codeexample}[]
\tikz [very thick] \draw [-{Stealth[] . Stealth[] Stealth[]}] (0,0) -- (2,0);
\end{codeexample}
Adding options to |.| is permissible, but they have no effect. In
particular, |sep| has no effect since a dot is not an arrow.




\subsubsection{Defining Shorthands}
\label{section-arrow-tip-macro}

It is often desirable to create ``shorthands'' for the names of arrow
tips that you are going to use very often. Indeed, in most documents
you will only need a single arrow tip kind and it would be useful that
you could refer to it just as |>| in your arrow tip specifications. As
another example, you might constantly wish to switch between a filled
and a non-filled circle as arrow tips and would like to use |*| and
|o| are shorthands for these case. Finally, you might just like to
shorten a long name like |Computer Modern Rightarrow| down to just,
say |To| or something similar.

All of these case can be addressed by defining appropriate
shorthands. This is done using the following handler:

\begin{handler}{{.tip}{=\meta{end specification}}}
  Defined the \meta{key} as a name that can be used inside arrow tip
  specifications. If the \meta{key} has a path before it, this path is
  ignored (so there is only one ``namespace'' for arrow
  tips). Whenever it is used, it will be replaced by the   \meta{end
    specification}. Note that you must \emph{always} provide
  (only) an end specification; when the \meta{key} is used inside a
  start specification, the ordering and the meaning of the keys inside
  the \meta{end specification} are translated automatically.

\begin{codeexample}[]
\tikz [foo /.tip = {Stealth[sep]. >>}]
  \draw [-foo] (0,0) -- (2,0);  
\end{codeexample}
\begin{codeexample}[]
\tikz [foo /.tip = {Stealth[sep] Latex[sep]},
       bar /.tip = {Stealth[length=10pt,open]}]
  \draw [-{foo[red] . bar}] (0,0) -- (2,0);  
\end{codeexample}

  In the last of the examples, we used |foo[red]| to make the arrows
  red. Any options given to a shorthand upon use will be passed on to
  the actual arrows tip for which the shorthand stands. Thus, we could
  also have written |Stealth[sep,red]| |Latex[sep,red]| instead of
  |foo[red]|. In other words, the ``replacement'' of a shorthand by
  its ``meaning'' is a semantic replacement rather than a syntactic
  replacement. In particular, the \meta{end specification} will be
  parsed immediately when the shorthand is being defined. However,
  this applies only to the options inside the specification, whose
  values are evaluated immediately. In contrast, which actual arrow
  tip kind is meant by a given shorthand used inside the \meta{end
    specification} is resolved only up each use of the shorthand. This
  means that when you write
  \begin{quote}
    |dup /.tip = >>|
  \end{quote}
  and then later write
  \begin{quote}
    |> /.tip = whatever|
  \end{quote}
  then |dup| will have the effect as if you had written
  |whatever[]whatever[]|. You will find that this behaviour is 
  what one would expect.

  There is one problem we have not yet addressed: The asymmetry of
  single letter arrow tips like |>| or |)|. When someone writes 
\begin{codeexample}[]
\tikz \draw [<->] (0,0) -- (1,0);  
\end{codeexample}
  we rightfully expect one arrow tip pointing left at the left end and
  an arrow tip pointing right at the right end. However, compare 
\begin{codeexample}[]
\tikz \draw [>->] (0,0) -- (1,0);  
\end{codeexample}
\begin{codeexample}[]
\tikz \draw [Stealth-Stealth] (0,0) -- (1,0);  
\end{codeexample}
  In both cases, we have \emph{identical} text in the start and end
  specifications, but in the first case we rightfully expect the left
  arrow to be flipped.

  The solution to this problem is that it is possible to define
  two names for the same arrow tip, namely one that is used inside
  start specifications and one for end specifications. Now, we can
  decree that the ``name of |>|'' inside start specifications is
  simply |<| and the above problems disappear.
  
  To specify different names for a shorthand in start and end
  specifications, use the following syntax: Instead of \meta{key}, you
  use \meta{name in start specifications}|-|\meta{name in end
    specifications}. Thus, to set the |>| key correctly, you actually
  need to write
\begin{codeexample}[]
\tikz [<-> /.tip = Stealth] \draw [<->>] (0,0) -- (1,0);  
\end{codeexample}
\begin{codeexample}[]
\tikz [<-> /.tip = Latex] \draw [>-<] (0,0) -- (1,0);  
\end{codeexample}

  Note that the above also works even though we have not set |<| as an
  arrow tip name for end specifications! The reason this works is that
  the \tikzname\ (more precisely, \pgfname) actually uses the
  following definition internally: 
  \begin{quote}
    |>-< /.tip = >[reversed]|
  \end{quote}
  Translation: ``When |<| is used in an end specification, please
  replace it by |>|, but reversed. Also, when |>| is used in a start
  specification, we also mean this inverted |>|.''
  
  By default, |>| is a shorthand for |To| and |To| is a shorthand for
  |to| (an arrow from the old libraries) when |arrows.meta| is not
  loaded library. When |arrow.meta| is loaded, |To| is redefined to
  mean the same as 
  |Computer Modern Rightarrow|.
\end{handler}

\begin{key}{/tikz/>=\meta{end arrow specification}}
  This is a short way of saying |<->/.tip=|\meta{end arrow specification}. 
\begin{codeexample}[]
\begin{tikzpicture}[scale=2,ultra thick]
  \begin{scope}[>=Latex]
    \draw[>->]    (0pt,3ex) -- (1cm,3ex);
    \draw[|<->>|] (0pt,2ex) -- (1cm,2ex);
  \end{scope}
  \begin{scope}[>=Stealth]
    \draw[>->]    (0pt,1ex) -- (1cm,1ex);
    \draw[|<<.<->|] (0pt,0ex) -- (1cm,0ex);
  \end{scope}
\end{tikzpicture}
\end{codeexample}

\end{key}



\subsubsection{Scoping of Arrow Keys}
\label{section-arrow-scopes}

There are numerous places where you can specify keys for an arrow
tip. There is, however, one final place that we have not yet
mentioned:

\begin{key}{/tikz/arrows=|[|\meta{arrow keys}|]|}
  The |arrows| key, which is normally used to set the arrow tips for
  the current scope, can also be used to set some arrow keys for the
  current scope. When the argument to |arrows| starts with an opening
  bracket and only otherwise contains one further closing bracket at
  the very end, this semantic of the |arrow| key is assumed.

  The \meta{arrow keys} will be set for the rest of current
  scope. This is useful for generally setting some design parameters
  or for generally switching on, say, bending as in:
\begin{codeexample}[code only]
\tikz [arrows={[bend]}] ... % Bend all arrows    
\end{codeexample}
\end{key}

We can now summarize which arrow keys are applied in what order when
an arrow tip is used:
\begin{enumerate}
\item First, the so-called \emph{defaults} are applied, which are
  values for the different parameters of a key. They are fixed in the
  definition of the key and cannot be changed. Since they are executed
  first, they are only the ultimate fallback.
\item The \meta{keys} from the use of |arrows=[|\meta{keys}|]| in all
  enclosing scopes.
\item Recursively, the \meta{keys} provided with the arrow
  tip inside shorthands.
\item The keys provided at the beginning of an arrow tip specification
  in brackets.
\item The keys provided directly next to the arrow tip inside the
  specification. 
\end{enumerate}




\subsection{Reference: Arrow Tips}
\label{section-arrows-meta}

\begin{pgflibrary}{arrows.meta}
  This library defines a large number of standard ``meta'' arrow
  tips. ``Meta'' means that you can configure these arrow tips in many
  different ways like changing their size or their line caps and joins
  and many other details.

  The only reason this library is not loaded by default is for
  compatibility with older versions of \tikzname. You can, however,
  safely load and use this library alongside the older libraries
  |arrows| and |arrows.spaced|.
\end{pgflibrary}


The different arrow tip kinds defined in the |arrows.meta| library can
be classified in different groups:
\begin{itemize}
\item \emph{Barbed} arrow tips consist mainly of lines that ``point
  backward'' from the tip of the arrow and which are not filled. For
  them, filling has no effect. A typical example is \tikz [baseline]
  \draw (0,.5ex) -- (1.5em,.5ex) [-Straight Barb];. Here is the list
  of defined arrow tips:

  \begin{arrowexamples}
    \arrowexample Arc Barb[]
    \arrowexample Bar[]
    \arrowexample Bracket[]
    \arrowexample Hooks[]
    \arrowexample Parenthesis[]
    \arrowexample Straight Barb[]
    \arrowexample Tee Barb[]
  \end{arrowexamples}
  
  All of these arrow tips can be configured and resized in many
  different ways as described in the following. Above, they are shown at
  their ``natural'' sizes, which are chosen in such a way that for a
  line width of 0.4pt their width matches the height
  of a letter ``x'' in Computer Modern at 11pt (with some
  ``overshooting'' to create visual consistency). 
\item \emph{Mathematical} arrow tips are actually a subclass of the
  barbed arrow tips, but we list them separately. They contain arrow
  tips that look exactly like the tips of arrows used in mathematical
  fonts such as the |\to|-symbol $\to$ from standard \TeX.
  \begin{arrowexamples}
    \arrowexample Classical TikZ Rightarrow[]
    \arrowexample Computer Modern Rightarrow[]
    \arrowexampledouble Implies[]
    \arrowexample To[]
  \end{arrowexamples}
  The |To| arrow tip is a shorthand for |Computer Modern Rightarrow|
  when |arrows.meta| is loaded.
\item \emph{Geometric} arrow tips consist of a filled shape like a
  kite or a circle or a ``stealth-fighter-like'' shape. A typical
  example is \tikz [baseline] \draw (0,.5ex) -- (1.5em,.5ex)
  [-Stealth];. These arrow tips can also be used in an ``open''
  variant as in \tikz [baseline] \draw (0,.5ex) -- (1.5em,.5ex)
  [-{Stealth[open]}];.

  \begin{arrowexamples}
    \arrowexample Circle[]
    \arrowexample Diamond[]
    \arrowexample Ellipse[]
    \arrowexample Kite[]
    \arrowexample Latex[]
    \arrowexample Latex[round]
    \arrowexample Rectangle[]
    \arrowexample Square[]
    \arrowexample Stealth[]
    \arrowexample Stealth[round]
    \arrowexample Triangle[]
    \arrowexample Turned Square[]
  \end{arrowexamples}
  
  Here are the ``open'' variants:
  
  \begin{arrowexamples}
    \arrowexample Circle[open]
    \arrowexample Diamond[open]
    \arrowexample Ellipse[open]
    \arrowexample Kite[open]
    \arrowexample Latex[open]
    \arrowexample Latex[round,open]
    \arrowexample Rectangle[open]
    \arrowexample Square[open]
    \arrowexample Stealth[open]
    \arrowexample Stealth[round,open]
    \arrowexample Triangle[open]
    \arrowexample Turned Square[open]
  \end{arrowexamples}
  
  Note that ``open'' arrow tips are not the same as ``filled with
  white,'' which is also available (just say |fill=white|). The
  difference is that the background will ``shine through'' an open
  arrow, while a filled arrow always obscures the background:
  
\begin{codeexample}[]
\tikz {
  \shade [left color=white, right color=red!50] (0,0) rectangle (4,1);

  \draw [ultra thick,-{Triangle[open]}]       (0,2/3) -- ++ (3,0);
  \draw [ultra thick,-{Triangle[fill=white]}] (0,1/3) -- ++ (3,0);
}
\end{codeexample}

\item \emph{Cap} arrow tips are used to add a ``cap'' to the end of a
  line. The graphic languages underlying \tikzname\ (\textsc{pdf},
  \textsc{postscript} or \textsc{svg}) all support three basic types
  of line caps on a very low level: round, rectangular, and ``butt.''
  Using cap arrow tips, you can add new caps to lines and use
  different caps for the end and the start. An example is the line
  \tikz [baseline] \draw [line width=1ex, {Round
    Cap[reversed]}-{Triangle Cap[] . Fast Triangle[] Fast Triangle[]}]
  (0,0.5ex) -- (2em,0.5ex);.
  \begin{arrowcapexamples}
    \arrowcapexample Butt Cap[]
    \arrowcapexample Fast Round[]
    \arrowcapexample Fast Triangle[]
    \arrowcapexample Round Cap[]
    \arrowcapexample Triangle Cap[]
  \end{arrowcapexamples}
\item \emph{Special} arrow tips are used for some specific purpose and
  do not fit into the above categories.
  \begin{arrowexamples}
    \arrowexample Rays[]
    \arrowexample Rays[n=8]
  \end{arrowexamples}
\end{itemize}

\subsubsection{Barbed Arrow Tips}

\begin{arrowtip}{Arc Barb}
  {
    This arrow tip attaches an arc to the end of the line whose angle
    is given by the |arc| option. The
    |length| and |width| parameters refer to the size of the arrow tip
    for |arc| set to 180 degrees, which is why in the example for 
    |arc=210| the actual length is larger than the specified
    |length|. The line width is taken into account 
    for the computation  of the length and width. Use the |round|
    option to add round caps to the end of the arcs.
  }
  {length=1.5cm,arc=210}
  {length=1.5cm,width=3cm}
  
  \begin{arrowexamples}
    \arrowexample[]
    \arrowexampledup[sep]
    \arrowexampledupdot[sep]
    \arrowexample[arc=120]
    \arrowexample[arc=270]
    \arrowexample[length=2pt]
    \arrowexample[length=2pt,width=5pt]
    \arrowexample[line width=2pt]
    \arrowexample[reversed]
    \arrowexample[round]
    \arrowexample[slant=.3]
    \arrowexample[left]
    \arrowexample[right]
    \arrowexample[harpoon,reversed]
    \arrowexample[red]
  \end{arrowexamples}
  The following options have no effect: |open|, |fill|.

  On |double| lines, the arrow tip will not look correct.
\end{arrowtip}


\begin{arrowtipsimple}{Bar}
  A simple bar. This is a simple instance of |Tee Barb| for length zero.
\end{arrowtipsimple}

\begin{arrowtip}{Bracket}
  {
    This is an instance of the |Tee Barb| arrow tip that results in
    something resembling a bracket. Just like the |Parenthesis| arrow
    tip, a |Bracket| is not modelled from a text square bracket, but
    rather its size has been chosen so that it fits with the other
    arrow tips.
  }
  {}
  {}
  
  \begin{arrowexamples}
    \arrowexample[]
    \arrowexampledup[sep]
    \arrowexampledupdot[sep]
    \arrowexample[reversed]
    \arrowexample[round]
    \arrowexample[slant=.3]
    \arrowexample[left]
    \arrowexample[right]
    \arrowexample[harpoon,reversed]
    \arrowexample[red]
  \end{arrowexamples}
  The following options have no effect: |open|, |fill|.

  On |double| lines, the arrow tip will not look correct.
\end{arrowtip}

\begin{arrowtip}{Hooks}
  {
    This arrow tip attaches two ``hooks'' to the end of the line. The
    |length| and |width| parameters refer to the size of the arrow tip
    if both arcs are 180 degrees; in the example the arc is 210
    degrees and, thus, the arrow is actually longer that the |length|
    dictates. The line width is taken into account for the computation
    of the length and width. The |arc| option is used to specify the
    angle of the arcs. Use the |round| option to add round caps to the
    end of the arcs.
  }
  {length=1cm,width=3.5cm,arc=210}
  {length=1cm,width=3.5cm}
  
  \begin{arrowexamples}
    \arrowexample[]
    \arrowexampledup[sep]
    \arrowexampledupdot[sep]
    \arrowexample[arc=120]
    \arrowexample[arc=270]
    \arrowexample[length=2pt]
    \arrowexample[length=2pt,width=5pt]
    \arrowexample[line width=2pt]
    \arrowexample[reversed]
    \arrowexample[round]
    \arrowexample[slant=.3]
    \arrowexample[left]
    \arrowexample[right]
    \arrowexample[harpoon,reversed]
    \arrowexample[red]
  \end{arrowexamples}
  The following options have no effect: |open|, |fill|.

  On |double| lines, the arrow tip will not look correct.
\end{arrowtip}


\begin{arrowtip}{Parenthesis}
  {
    This arrow tip is an instantiation of the |Arc Barb| so that it
    resembles a parenthesis. However, the idea is not to recreate a
    ``real'' parenthesis as it is used in text, but rather a ``bow''
    at a size that harmonizes with the other arrow tips at their
    default sizes.
  }
  {}
  {}
  
  \begin{arrowexamples}
    \arrowexample[]
    \arrowexampledup[sep]
    \arrowexampledupdot[sep]
    \arrowexample[reversed]
    \arrowexample[round]
    \arrowexample[slant=.3]
    \arrowexample[left]
    \arrowexample[right]
    \arrowexample[harpoon,reversed]
    \arrowexample[red]
  \end{arrowexamples}
  The following options have no effect: |open|, |fill|.

  On |double| lines, the arrow tip will not look correct.
\end{arrowtip}

\begin{arrowtip}{Straight Barb}
  {
    This is the ``archetypal'' arrow head, consisting of just two
    straight lines. The |length| and |width| parameters refer to the
    horizontal and vertical distances between the points on the path
    making up the arrow tip. As can be seen, the line width of the
    arrow tip's path is not taken into account. 
    The |angle| option is particularly useful to set
    the opening angle at the tip of the arrow head. The |round| option
    gives a ``softer'' or ``rounder'' version of the arrow tip. 
  }
  {length=2cm,width=3cm}
  {length=2cm/-4mm,width=3cm}
  
  \begin{arrowexamples}
    \arrowexample[]
%    \arrowexampledouble[]
    \arrowexampledup[]
    \arrowexampledupdot[]
    \arrowexample[length=5pt]
    \arrowexample[length=5pt,width=5pt]
    \arrowexample[line width=2pt]
    \arrowexample[reversed]
    \arrowexample[angle=60:2pt 3]
    \arrowexample[round]
    \arrowexample[slant=.3]
    \arrowexample[left]
    \arrowexample[right]
    \arrowexample[harpoon,reversed]
    \arrowexample[red]
  \end{arrowexamples}
  The following options have no effect: |open|, |fill|.

  On |double| lines, the arrow tip will not look correct.
\end{arrowtip}


\begin{arrowtip}{Tee Barb}
  {
    This arrow tip attaches a little ``T'' on both sides of the
    tip. The arrow |inset| dictates the distance from the back end to
    the middle of the stem of the T. When the inset is equal to the
    length, the arrow tip is drawn as a single line, not as three
    lines (this is important for the ``round'' version since, then,
    the corners get rounded). 
  }
  {length=1.5cm,width=3cm,inset=1cm}
  {length=1.5cm,width=3cm,inset=1cm}
  
  \begin{arrowexamples}
    \arrowexample[]
    \arrowexampledup[sep]
    \arrowexampledupdot[sep]
    \arrowexample[inset=0pt]
    \arrowexample[inset'=0pt 1]
    \arrowexample[line width=2pt]
    \arrowexample[round]
    \arrowexample[round,inset'=0pt 1]
    \arrowexample[slant=.3]
    \arrowexample[left]
    \arrowexample[right]
    \arrowexample[harpoon,reversed]
    \arrowexample[red]
  \end{arrowexamples}
  The following options have no effect: |open|, |fill|.

  On |double| lines, the arrow tip will not look correct.
\end{arrowtip}



\subsubsection{Mathematical Barbed Arrow Tips}


\begin{arrowtip}{Classical TikZ Rightarrow}
  {
    This arrow tip is the ``old'' or ``classical'' arrow tip that used
    to be the standard in \tikzname\ in earlier versions. It was
    modelled on an old version of the tip of \texttt{\string\rightarrow}
    ($\rightarrow$) of the Computer Modern fonts. However, this ``old
    version'' was really old, Donald 
    Knuth (the designer of both \TeX\ and of the Computer Modern
    fonts) replaced the arrow tip of the mathematical fonts
    in~1992. 
  }
  {length=1cm,width=2cm}
  {length=1cm,width=2cm}
  The main problem with this arrow tip is that it is ``too small''
  at its natural size. I recommend using the new \texttt{Computer Modern
    Rightarrow} arrow tip instead, which matches the current
  $\to$. This new version is also the default used as |>| and as
  |To|, now.
  
  \begin{arrowexamples}
    \arrowexample[]
    \arrowexampledup[sep]
    \arrowexampledupdot[sep]
    \arrowexample[length=3pt]
    \arrowexample[sharp]
    \arrowexample[slant=.3]
    \arrowexample[left]
    \arrowexample[right]
    \arrowexample[harpoon,reversed]
    \arrowexample[red]
  \end{arrowexamples}
  The following options have no effect: |open|, |fill|.

  On |double| lines, the arrow tip will not look correct.
\end{arrowtip}

\begin{arrowtip}{Computer Modern Rightarrow}
  {
    For a line width of 0.4pt (the default), this arrow tip looks very
    much like \texttt{\string\rightarrow} ($\to$) of the Computer Modern math
    fonts. However, it is not a ``perfect'' match: the line caps and
    joins of the ``real'' $\to$ are rounded differently from this
    arrow tip; but it takes a keen eye to notice the difference. When
    the |arrow.meta| library is loaded, this arrow tip becomes the
    default of |To| and, thus, is used whenever |>| is used (unless,
    of course, you redefined |>|).
  }
  {length=1cm,width=2cm}
  {length=1cm,width=2cm}
  
  \begin{arrowexamples}
    \arrowexample[]
    \arrowexampledup[sep]
    \arrowexampledupdot[sep]
    \arrowexample[length=3pt]
    \arrowexample[sharp]
    \arrowexample[slant=.3]
    \arrowexample[left]
    \arrowexample[right]
    \arrowexample[harpoon,reversed]
    \arrowexample[red]
  \end{arrowexamples}
  The following options have no effect: |open|, |fill|.

  On |double| lines, the arrow tip will not look correct.
\end{arrowtip}



\begin{arrowtipsimple}{Implies}
  This arrow tip makes only sense in conjunction with the |double|
  option. The idea is that you attach it to a double line to get
  something that looks like \TeX's \texttt{\string\implies} arrow
  ($\implies$). A typical use of this arrow tip is
\begin{codeexample}[]
\tikz \graph [clockwise=3, math nodes,
              edges = {double equal sign distance, -Implies}] {
  "\alpha", "\beta", "\gamma";
  "\alpha" -> "\beta" -> "\gamma" -> "\alpha"
};
\end{codeexample}
  \begin{arrowexamples}
    \arrowexampledouble[]
    \arrowexampledouble[red]
  \end{arrowexamples}
\end{arrowtipsimple}

\begin{arrowtipsimple}{To}
  This is a shorthand for  |Computer Modern Rightarrow| when the
  |arrow.meta| library is loaded. Otherwise, it is a shorthand for the
  classical \tikzname\ rightarrow.
\end{arrowtipsimple}




\subsubsection{Geometric Arrow Tips}



\begin{arrowtip}{Circle}
  {
    Although this tip is called ``circle,'' you can also use it to
    draw ellipses if you set the length and width to different
    values. Neither |round| nor |reversed| has any effect on this
    arrow tip. 
  }
  {length=2cm,width=2cm}
  {length=2cm,width=2cm}
  
  \begin{arrowexamples}
    \arrowexample[]
    \arrowexampledup[sep]
    \arrowexampledupdot[sep]
    \arrowexample[open]
    \arrowexample[length=3pt]
    \arrowexample[slant=.3]
    \arrowexample[left]
    \arrowexample[right]
    \arrowexample[red]
  \end{arrowexamples}
\end{arrowtip}

\begin{arrowtipsimple}{Diamond}
  This is an instance of |Kite| where the length is larger than the
  width. 

  \begin{arrowexamples}
    \arrowexample[]
    \arrowexampledup[]
    \arrowexampledupdot[]
    \arrowexample[open]
    \arrowexample[length=10pt]
    \arrowexample[round]
    \arrowexample[slant=.3]
    \arrowexample[left]
    \arrowexample[right]
    \arrowexample[red]
    \arrowexample[fill=red!50]
  \end{arrowexamples}
\end{arrowtipsimple}

\begin{arrowtipsimple}{Ellipse}
  This is a shorthand for a ``circle'' that is twice as wide as high.
  
  \begin{arrowexamples}
    \arrowexample[]
    \arrowexampledup[sep]
    \arrowexampledupdot[sep]
    \arrowexample[open]
    \arrowexample[length=10pt]
    \arrowexample[round]
    \arrowexample[slant=.3]
    \arrowexample[left]
    \arrowexample[right]
    \arrowexample[red]
    \arrowexample[fill=red!50]
  \end{arrowexamples}
\end{arrowtipsimple}


\begin{arrowtip}{Kite}
  {
    This arrow tip consists of four lines that form a ``kite.'' The
    |inset| prescribed how far the width-axis of the kite is removed
    from the back end. Note that the inset cannot be negative, use a
    |Stealth| arrow tip for this.
  }
  {length=3cm,width=2cm,inset=1cm}
  {length=3cm,width=2cm,inset=1cm}
  
  \begin{arrowexamples}
    \arrowexample[]
    \arrowexampledup[sep]
    \arrowexampledupdot[sep]
    \arrowexample[open]
    \arrowexample[length=6pt,width=4pt]
    \arrowexample[length=6pt,width=4pt,inset=1.5pt]
    \arrowexample[round]
    \arrowexample[slant=.3]
    \arrowexample[left]
    \arrowexample[right]
    \arrowexample[red]
  \end{arrowexamples}
\end{arrowtip}


\begin{arrowtip}{Latex}
  {
    This arrow tip is the same as the arrow tip used in \LaTeX's
    standard pictures (via the \texttt{\string\vec} command), if you
    set the length to 4pt. The default size for this arrow tip was set
    slightly larger so that it fits better with the other geometric
    arrow tips. 
  }
  {length=3cm,width=2cm}
  {length=3cm,width=2cm}
  
  \begin{arrowexamples}
    \arrowexample[]
    \arrowexampledup[sep]
    \arrowexampledupdot[sep]
    \arrowexample[open]
    \arrowexample[length=4pt]
    \arrowexample[round]
    \arrowexample[slant=.3]
    \arrowexample[left]
    \arrowexample[right]
    \arrowexample[red]
  \end{arrowexamples}
\end{arrowtip}

\begin{arrowtipsimple}{LaTeX}
  Another spelling for the |Latex| arrow tip.
\end{arrowtipsimple}



\begin{arrowtip}{Rectangle}
  {
    A rectangular arrow tip. By default, it is twice as long as high. 
  }
  {length=3cm,width=2cm}
  {length=3cm,width=2cm}
  
  \begin{arrowexamples}
    \arrowexample[]
    \arrowexampledup[sep]
    \arrowexampledupdot[sep]
    \arrowexample[open]
    \arrowexample[length=4pt]
    \arrowexample[round]
    \arrowexample[slant=.3]
    \arrowexample[left]
    \arrowexample[right]
    \arrowexample[red]
  \end{arrowexamples}
\end{arrowtip}


\begin{arrowtipsimple}{Square}
  An instance of the |Rectangle| whose width is identical to the length.
  
  \begin{arrowexamples}
    \arrowexample[]
    \arrowexampledup[sep]
    \arrowexampledupdot[sep]
    \arrowexample[open]
    \arrowexample[length=4pt]
    \arrowexample[round]
    \arrowexample[slant=.3]
    \arrowexample[left]
    \arrowexample[right]
    \arrowexample[red]
  \end{arrowexamples}
\end{arrowtipsimple}



\begin{arrowtip}{Stealth}
  {
    This arrow tip is similar to a |Kite|, only the |inset| now counts
    ``inwards.'' Because of that sharp angles, for this arrow tip is
    makes quite a difference, visually, if use the |round|
    option. Also, using the |harpoon| option (or |left| or |right|)
    will \emph{lengthen} the arrow tip because of the even sharper
    corner at the tip.
  }
  {length=3cm,width=2cm,inset=1cm}
  {length=3cm,width=2cm,inset=1cm}
  
  \begin{arrowexamples}
    \arrowexample[]
    \arrowexampledup[sep]
    \arrowexampledupdot[sep]
    \arrowexample[open]
    \arrowexample[length=6pt,width=4pt]
    \arrowexample[length=6pt,width=4pt,inset=1.5pt]
    \arrowexample[round]
    \arrowexample[slant=.3]
    \arrowexample[left]
    \arrowexample[right]
    \arrowexample[red]
  \end{arrowexamples}
\end{arrowtip}


\begin{arrowtipsimple}{Triangle}
  An instance of a |Kite| with zero inset.
  
  \begin{arrowexamples}
    \arrowexample[]
    \arrowexampledup[sep]
    \arrowexampledupdot[sep]
    \arrowexample[open]
    \arrowexample[length=4pt]
    \arrowexample[angle=45:1pt 3]
    \arrowexample[angle=60:1pt 3]
    \arrowexample[angle=90:1pt 3]
    \arrowexample[round]
    \arrowexample[slant=.3]
    \arrowexample[left]
    \arrowexample[right]
    \arrowexample[red]
  \end{arrowexamples}
\end{arrowtipsimple}

\begin{arrowtipsimple}{Turned Square}
  An instance of a |Kite| with identical width and height and mid-inset.
  
  \begin{arrowexamples}
    \arrowexample[]
    \arrowexampledup[sep]
    \arrowexampledupdot[sep]
    \arrowexample[open]
    \arrowexample[length=4pt]
    \arrowexample[round]
    \arrowexample[slant=.3]
    \arrowexample[left]
    \arrowexample[right]
    \arrowexample[red]
  \end{arrowexamples}
\end{arrowtipsimple}



\subsubsection{Caps}

Recall that a \emph{cap} is a way of ending a line. The graphic
languages underlying \tikzname\ (\textsc{pdf}, \textsc{postscript} or
\textsc{svg}) all support three basic types of line caps on a very low
level: round, rectangular, and ``butt.'' Using cap arrow tips, you can
add new caps to lines and use different caps for the end and the
start.


\begin{arrowtipsimple}{Butt Cap}
  This arrow tip ends the line ``in the normal way'' with a straight
  end. This arrow tip is only need to ``cover up'' the actual line
  cap, if this happens to differ from the normal cap. In the following
  example, the line cap is ``round'', but, nevertheless, the right end
  is a ``butt'' cap:

\begin{codeexample}[]
\tikz \draw [line width=1ex, line cap=round, -Butt Cap] (0,0) -- (1,0);    
\end{codeexample}
\end{arrowtipsimple}


\begin{arrowcap}{Fast Round}
  {
    This arrow tip is not really a cap, you use it in conjunction with
    (typically) the |Round Cap|. The idea is that you end your line
    using the round cap and then add several \texttt{Fast
      Round}s. As for |Round Cap|, the |length| parameter
    dictates the length is the length of the ``main part,'' the
    inset sets the length of a line that comes before this tip.
  }
  {length=5mm,inset=1cm}
  {length=5mm,inset=-1cm}
  {-15mm}
  
\begin{codeexample}[]
\tikz \draw [line width=1ex,
             -{Round Cap []. Fast Round[] Fast Round[]}]
  (0,0) -- (1,0);             
\end{codeexample}
  Note that in conjunction with the |bend| option, this works even
  quite well for curves:
\begin{codeexample}[]
\tikz [f/.tip = Fast Round] % shorthand
  \draw [line width=1ex, -{[bend] Round Cap[] . f f f}]
  (0,0) to [bend left] (1,0);             
\end{codeexample}
  
  \begin{arrowcapexamples}
    \arrowcapexample[]
    \arrowcapexample[reversed]
    \arrowcapexample[cap angle=60]
    \arrowcapexample[cap angle=60,inset=5pt]
    \arrowcapexample[length=.5ex]
    \arrowcapexample[slant=.3]
  \end{arrowcapexamples}
\end{arrowcap}


\begin{arrowcap}{Fast Triangle}
  {
    This arrow tip works like |Fast Round|, only for triangular caps. 
  }
  {length=5mm,inset=1cm}
  {length=5mm,inset=-1cm}
  {-15mm}
  
\begin{codeexample}[]
\tikz \draw [line width=1ex,
             -{Triangle Cap []. Fast Triangle[] Fast Triangle[]}]
  (0,0) -- (1,0);             
\end{codeexample}
  Again, this tip works well for curves:
\begin{codeexample}[]
\tikz [f/.tip = Fast Triangle] % shorthand
  \draw [line width=1ex, -{[bend] Triangle Cap[] . f f f}]
  (0,0) to [bend left] (1,0);             
\end{codeexample}
  
  \begin{arrowcapexamples}
    \arrowcapexample[]
    \arrowcapexample[reversed]
    \arrowcapexample[cap angle=60]
    \arrowcapexample[cap angle=60,inset=5pt]
    \arrowcapexample[length=.5ex]
    \arrowcapexample[slant=.3]
  \end{arrowcapexamples}
\end{arrowcap}



\begin{arrowcap}{Round Cap}
  {
    This arrow tip ends the line using a half circle or, if the length
    has been modified, a half-ellipse.
  }
  {length=5mm}
  {length=5mm}
  {-5mm}
  \begin{arrowcapexamples}
    \arrowcapexample[]
    \arrowcapexample[reversed]
    \arrowcapexample[length=.5ex]
    \arrowcapexample[slant=.3]
  \end{arrowcapexamples}
\end{arrowcap}



\begin{arrowcap}{Triangle Cap}
  {
    This arrow tip ends the line using a triangle whose length is
    given by the |length| option.
  }
  {length=5mm}
  {length=5mm}
  {-5mm}

  You can get any angle you want at the tip by specifying a length
  that is an appropriate multiple of the line width. The following
  options does this computation for you:
  \begin{key}{/pgf/arrow keys/cap angle=\meta{angle}}
    Sets |length| to an appropriate multiple of the line width so that
    the angle of a |Triangle Cap| is exactly \meta{angle} at the tip.
  \end{key}
  
  \begin{arrowcapexamples}
    \arrowcapexample[]
    \arrowcapexample[reversed]
    \arrowcapexample[cap angle=60]
    \arrowcapexample[cap angle=60,reversed]
    \arrowcapexample[length=.5ex]
    \arrowcapexample[slant=.3]
  \end{arrowcapexamples}
\end{arrowcap}


\subsubsection{Special Arrow Tips}

\begin{arrowtip}{Rays}
  {
    This arrow tip attaches a ``bundle of rays'' to the tip. The
    number of evenly spaced rays is given by the |n| arrow key (see
    below). When the number is even, the rays will lie to the left and
    to the right of the direction of the arrow; when the number is
    odd, the rays are rotated in such a way that one of them points
    perpendicular to the direction of the arrow (this is to ensure
    that no ray points in the direction of the line, which would look
    strange). The |length| and |width| describe the length and width
    of an ellipse into which the rays fit.
  }
  {length=3cm,width=3cm,n=6}
  {length=3cm,width=3cm}
  
  \begin{arrowexamples}
    \arrowexample[]
    \arrowexampledup[sep]
    \arrowexampledupdot[sep]
    \arrowexample[width'=0pt 2]
    \arrowexample[round]
    \arrowexample[n=2]
    \arrowexample[n=3]
    \arrowexample[n=4]
    \arrowexample[n=5]
    \arrowexample[n=6]
    \arrowexample[n=7]
    \arrowexample[n=8]
    \arrowexample[n=9]
    \arrowexample[slant=.3]
    \arrowexample[left]
    \arrowexample[right]
    \arrowexample[left,n=5]
    \arrowexample[right,n=5]
    \arrowexample[red]
  \end{arrowexamples}
\end{arrowtip}
\begin{key}{/pgf/arrow keys/n=\meta{number} (initially 4)}
  Sets the number of rays in a |Rays| arrow tip.
\end{key}


\endinput


